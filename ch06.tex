%%%%%%%%%%%%%%%%%%%%%%%%%%%%%%%%%%%%%%%%%%%%%%%%%%%%%%%%%%%%%%%%%%%%%%%%%%
%% DISC
%%%%%%%%%%%%%%%%%%%%%%%%%%%%%%%%%%%%%%%%%%%%%%%%%%%%%%%%%%%%%%%%%%%%%%%%%%

\chapter{Discrimants and Norms}\label{discnorm}

In this chapter we give a geometric interpretation of the discriminant
of an order in a number field. We also define norms of ideals and
prove that the norm function is multiplicative.  Discriminants of
orders and norms of ideals will play a crucial role in our proof of
finiteness of the class group in the next chapter.

\section{Viewing $\sO_K$ as a Lattice in a Real Vector Space}

Let~$K$ be a number field of degree $n$.  By the primitive element
theorem, $K=\QQ(\alpha)$ for some~$\alpha$, so we can write $K\isom
\QQ[x]/(f)$, where $f\in\QQ[x]$ is the minimal polynomial of~$\alpha$.
Because $\CC$ is algebraically closed and~$f$ is irreducible, it has
exactly $n=[K:\QQ]$ complex roots.  Each of these roots $z\in\CC$
induces a homomorphism $\QQ[x] \to \CC$ given by $x\mapsto z$, whose
kernel is the ideal $(f)$.  Thus we obtain~$n$ embeddings of $K\isom
\QQ[x]/(f)$ into~$\CC$: $\sigma_1,\dots, \sigma_n:K \hookrightarrow \CC$.
\begin{example}
  We compute the embeddings listed above for $K=\QQ(\sqrt[3]{2})$.
\begin{sagecode}
\begin{sagecell}
K = QQ[2^(1/3)]; K
\end{sagecell}
\begin{sageout}
Number Field in a with defining polynomial x^3 - 2
\end{sageout}
\begin{sagecell}
K.complex_embeddings()
\end{sagecell}
\begin{sageout}
[Ring morphism: ...
Defn: a |--> -0.629960524947 - 1.09112363597*I,
Ring morphism: ...
Defn: a |--> -0.629960524947 + 1.09112363597*I,
Ring morphism: ...
Defn: a |--> 1.25992104989]
\end{sageout}
\end{sagecode}
\end{example}


Let $\sigma:K \hookrightarrow \CC^n$ be the map $a\mapsto
(\sigma_1(a),\dots,\sigma_n(a))$, and let $V=\RR\sigma(K)$ be the
$\RR$-span of the image $\sigma(K)$ of~$K$ inside $\CC^n$.

\begin{lemma}\label{lem:disc_finite}
  Suppose $L\subset \RR^n$ is a subgroup of the vector space~$\RR^n$.
  Then the induced topology on~$L$ is discrete if and only
  if for every  $H>0$ the set
  \[
    X_H = \{v \in L : \max\{|v_1|,\dots,|v_n|\} \leq H \}
  \]
  is finite.
\end{lemma}
\begin{proof}
  If~$L$ is not discrete, then there is a point $x \in L$ such that
  for every $\varepsilon > 0$ there is $y \in L$ such that
  $0 < |x-y| < \varepsilon$. By choosing smaller and smaller~$\varepsilon$,
  we find infinitely many elements $x-y\in L$
  all of whose coordinates are smaller than~$1$.
  The set $X_1$ is thus not finite.   Thus if the sets
  $X_H$ are all finite,~$L$ must be discrete.

  Next assume that~$L$ is discrete and let $H>0$ be any positive number.
  Then for every $x\in X_H$ there is an open ball $B_x$ that
  contains~$x$ but no other element of~$L$.  Since $X_H$ is closed and
  bounded, the Heine-Borel theorem implies that $X_H$ is compact, so the
  open covering $\cup B_x$ of $X_H$ has a finite subcover, which
  implies that $X_H$ is finite, as claimed.
\end{proof}

\begin{lemma}\label{lem:disc_rankdim}
  If~$L$ if a free abelian group that is
  discrete in a finite-dimensional
  real vector space~$V$ and $\RR{}L=V$, then the rank of~$L$
  equals the dimension of~$V$.
\end{lemma}
\begin{proof}
  Let $x_1,\dots, x_m \in L$ be an $\RR$-vector space basis for
  $\RR{}L$, and consider the $\ZZ$-submodule $M=\ZZ x_1 + \cdots + \ZZ
  x_m$ of~$L$.  If the quotient $L/M$ is infinite, then there are
  infinitely many distinct elements of~$L$ that all lie in a
  fundamental domain for~$M$, so Lemma~\ref{lem:disc_finite} implies
  that $L$ is not discrete.  This is a contradiction, so $L/M$ is
  finite, and the rank of~$L$ is $m=\dim(\RR L)$, as claimed.
\end{proof}

\begin{proposition}\iprop{dimension of embedding of field}
  The $\RR$-vector space~$V=\RR\sigma(K)$ spanned by the image
  $\sigma(K)$ of $K$ has dimension~$n$.
\end{proposition}
\begin{proof}
  We prove this by showing that the image $\sigma(\sO_K)$ is discrete. If
  $\sigma(\sO_K)$ were not discrete it would contain elements all of
  whose coordinates are simultaneously arbitrarily small.  The norm of
  an element $a\in \sO_K$ is the product of the entries of $\sigma(a)$,
  so the norms of nonzero elements of $\sO_K$ would go to~$0$.  This is a
  contradiction, since the norms of nonzero elements of $\sO_K$ are
  nonzero integers.

  Since $\sigma(\sO_K)$ is discrete in $\CC^n$, Lemma~\ref{lem:disc_rankdim}
  implies that $\dim(V)$ equals the rank of $\sigma(\sO_K)$.  Since~$\sigma$
  is injective, $\dim(V)$ is the rank of $\sO_K$, which equals~$n$ by
  Proposition~\ref{prop:ok_lattice}.
\end{proof}

\subsection{A Determinant}
Suppose $w_1, \dots, w_n$ is a basis for
$\sO_K$, and let $A$ be the matrix whose $i$th row is $\sigma(w_i)$.
Consider the determinant $\det(A)$.
\begin{example}
  The ring $\sO_K=\ZZ[i]$ of integers of $K=\QQ(i)$
  has $\ZZ$-basis $w_1=1$, $w_2=i$.
  The map $\sigma:K\to \CC^2$ is given by
  \[
    \sigma(a+bi) = (a+bi,a-bi)\in\CC^2.
  \]
  The image $\sigma(\sO_K)$ is spanned by
  $(1,1)$ and $(i,-i)$.
  The determinant is
  \[
    \left|\begin{pmatrix} 1 & 1 \\ i & -i \end{pmatrix}\right| = -2i.
  \]

  Let $\sO_K=\ZZ[\sqrt{2}]$ be the ring of integers of $K=\QQ(\sqrt{2})$.
  The map $\sigma$ is
  \[
    \sigma(a+b\sqrt{2}) = (a+b\sqrt{2},a-b\sqrt{2})\in\RR^2,
  \]
  and
  \[
    A = \begin{pmatrix} 1 & 1 \\ \sqrt{2} & -\sqrt{2} \end{pmatrix},
  \]
  which has determinant
  $ -2\sqrt{2}$.
\end{example}
As the above example illustrates, the determinant $\det(A)$ most
certainly need not be an integer.  However, as we will see, it's
square is an integer that does not depend on our choice of
basis for $\sO_K$.

\section{Discriminants}\label{sec:disc}

Suppose $w_1,\dots, w_n$ is a basis for $\sO_K$ as a $\ZZ$-module,
which we view as a $\QQ$-vector space.  Let $\sigma: K \hookrightarrow \CC^n$ be
the embedding $\sigma(a)=(\sigma_1(a),\dots,\sigma_n(a))$, where
$\sigma_1,\dots, \sigma_n$ are the distinct embeddings of $K$
into~$\CC$.  Let $A$ be the matrix whose rows are $\sigma(w_1), \dots,
\sigma(w_n)$.

Changing our choice of
basis for $\sO_K$ is the same as left multiplying~$A$ by an integer
matrix $U$ of determinant $\pm 1$, which changes
$\det(A)$ by $\pm 1$.
This leads us to consider $\det(A)^2$ instead, which does not depend
on the choice of basis; moreover, as we will see, $\det(A)^2$ is an integer.
Note that
\begin{align*}
  \det(A)^2 &= \det(AA) =
  \det(A)\det(A) = \det(A)\det(A^t) =
  \det(A A^t) \\
  &= \det\left(\sum_{k=1,\dots,n} \sigma_k(w_i)\sigma_k(w_j)\right)
  = \det\left(\sum_{k=1,\dots,n} \sigma_k(w_i w_j)\right)\\
  &= \det(\Trace(w_i w_j)_{1\leq i,j\leq n}),
\end{align*}
so $\det(A)^2$ can be defined purely in terms of the trace without
mentioning the embeddings $\sigma_i$.
Moreover, if we change basis hence multiplying $A$ by some $U$ with determinant $\pm 1$, then
$\det(UA)^2 = \det(U)^2\det(A)^2 = \det(A)^2$.
Because $\det(A)$ is an algebraic integer and $\Trace(w_iw_j)\in\QQ$, it follows
that $\det(A)^2$ is an algebraic integer in $\QQ$.
Thus $\det(A)^2\in \ZZ$ is well defined as a quantity associated to $\sO_K$.

If we view~$K$ as a $\QQ$-vector space, then $(x,y)\mapsto \Trace(xy)$
defines a bilinear pairing $K\times K \to \QQ$ on~$K$, which we call
the \defn{trace pairing}.  The following lemma asserts that this
pairing is nondegenerate, so $\det(\Trace(w_i w_j))\neq 0$ hence
$\det(A)\neq 0$.
\begin{lemma}\label{lem:tracenondegen}\ilem{trace pairing nondegenerate}
  The trace pairing is nondegenerate.
\end{lemma}
\begin{proof}
  If the trace pairing is degenerate, then there exists $0\neq a\in K$ such
  that for every $b\in K$ we have $\Trace(ab)=0$.  In particularly, taking
  $b=a^{-1}$ we see that $0=\Trace(a a^{-1})=\Trace(1)=[K:\QQ]>0$, which is
  absurd.
\end{proof}

\begin{definition}[Discriminant]\label{def:disc}
  Suppose $a_1,\dots, a_n$ is any $\QQ$-basis of $K$.  The \defn{discriminant}
  of $a_1,\dots, a_n$ is
  \[
    \Disc(a_1,\dots,a_n) = \det(\Trace(a_i a_j)_{1\leq i,j\leq n})\in\QQ.
  \]
  The \defn{discriminant} $\Disc(\sO)$ of an order $\sO$ in $\sO_K$ is
  the discriminant of any $\ZZ$-basis for~$\sO$.
  The \defn{discriminant} $d_K=\Disc(K)$ of the number field~$K$
  is the discriminant of $\sO_K$.
  Note that these discriminants are all nonzero
  by Lemma~\ref{lem:tracenondegen}.
\end{definition}

\begin{remark}
  It is also standard to define the discriminant of a monic polynomial
  to be the product of the differences of the roots.  If $\alpha\in
  \sO_K$ with $\ZZ[\alpha]$ of finite index in $\sO_K$, and~$f$ is the
  minimal polynomial of $\alpha$, then $\Disc(f)=\Disc(\ZZ[\alpha])$.
  To see this, note that if we choose the basis
  $1,\alpha,\dots,\alpha^{n-1}$ for $\ZZ[\alpha]$, then both
  discriminants are the square of the same Vandermonde determinant.
\end{remark}

\begin{remark}
  If $S/R$ is an extension of Dedekind domains, with $S$ a free $R$
  module of finite rank, then the above definition of a {\em relative}
  discriminant of $S/R$ does not make sense in general.  The problem
  is that $R$ may have more units than $\{\pm 1\}$, in which case
  $\det(A^2)$ is not well defined.   To generalize the notion of
  discriminant to arbitrary finite extensions of Dedekind domains,
  one must instead introduce a discriminant {\em ideal}.
\end{remark}

\begin{example}
  In {\Sage}, we compute the discriminant of a number field or order
  using the discriminant command:
\begin{sagecode}
\begin{sagecell}
K.<a> = NumberField(x^2 - 5)
K.discriminant()
\end{sagecell}
\begin{sageout}
5
\end{sageout}
\end{sagecode} %link
  \noindent{}This also works for orders (notice the square factor
  below, which will be explained
  by Proposition~\ref{prop:indsquare}):
\begin{sagecode} %link
\begin{sagecell}
R = K.order([7*a]); R
\end{sagecell}
\begin{sageout}
Order in Number Field in a with defining polynomial x^2 - 5
\end{sageout}
\begin{sagecell}
factor(R.discriminant())
\end{sagecell}
\begin{sageout}
2^2 * 5 * 7^2
\end{sageout}
\end{sagecode}
\end{example}

\begin{warning}
  In {\tt Magma} $\Disc(K)$ is defined to be the
  discriminant of the polynomial you happened to use to define~$K$.
\begin{sagecode}
\begin{lstlisting}[language=bash]
> K := NumberField(x^2-5);
> Discriminant(K);
20
\end{lstlisting}
\end{sagecode}
  This is an intentional choice done for efficiency reasons, since
  computing the maximal order can take a long time.  Nonetheless, it
  conflicts with standard mathematical usage, so beware.
\end{warning}

The following proposition asserts that the discriminant of an order
$\sO$ in $\sO_K$ is bigger than $\Disc(\sO_K)$ by a factor of the square
of the index.
\begin{proposition}\iprop{discriminant of order}\label{prop:indsquare}
  Suppose $\sO$ is an order in $\sO_K$. Then
  \[
    \Disc(\sO) =  \Disc(\sO_K)\cdot [\sO_K:\sO]^2.
  \]
\end{proposition}
\begin{proof}
  Let $A$ be a matrix whose rows are the images via $\sigma$ of a basis
  for $\sO_K$, and let $B$ be a matrix whose rows are the images via
  $\sigma$ of a basis for $\sO$.  Since $\sO\subset \sO_K$ has finite
  index, there is an integer matrix $C$ such that $CA=B$,
  and $\left|\det(C)\right|= [\sO_K:\sO]$.  Then
  \[\Disc(\sO) = \det(B)^2 = \det(CA)^2 = \det(C)^2\det(A)^2
  = [\sO_K:\sO]^2 \cdot \Disc(\sO_K).
  \]
\end{proof}

\begin{example}\label{ex:exdim}
  Let $K$ be a number field and consider the quantity
  \[
    D(K) = \gcd\{
    \Disc(\alpha) : \alpha \in \sO_K \text{ and } [\sO_K:\ZZ{}[\alpha]] < \infty
    \}.
  \]
  One might hope that $D(K)$ is equal to the discriminant $\Disc(\sO_K)$
  of $K$, but this is not the case in general.  Recall
  Example~\ref{ex:dedekind}, in which we considered the field $K$ generated
  by a root of $f = x^3 + x^2 - 2x+8$.  In that example, the
  discriminant of $\sO_K$ is $-503$ with $503$ prime:
\begin{sagecode}
\begin{sagecell}
K.<a> = NumberField(x^3 + x^2 - 2*x + 8)
factor(K.discriminant())
\end{sagecell}
\begin{sageout}
-1 * 503
\end{sageout}
\end{sagecode}
  For every $\alpha\in\sO_K$, we have $2\mid [\sO_K:\ZZ[\alpha]]$, since
  $\sO_K$ fails to be monogenic at $2$.  By
  Proposition~\ref{prop:indsquare}, the discriminant of $\ZZ[\alpha]$ is
  divisible by~$4$ for all~$\alpha$, so $\Disc(\alpha)$ is also
  divisible by~$4$.  This is why $2$ is called an ``inessential
  {\em discriminant} divisor''.
\end{example}

Proposition~\ref{prop:indsquare} gives an algorithm for computing $\sO_K$,
albeit a  slow one.  Given~$K$, find some order $\sO\subset
K$, and compute $d=\Disc(\sO)$.  Factor~$d$, and use the factorization
to write $d=s\cdot f^2$, where $f^2$ is the largest square that
divides~$d$.  Then the index of $\sO$ in $\sO_K$ is a divisor of~$f$,
and we (tediously) can enumerate all rings~$R$ with $\sO\subset
R\subset K$ and $[R:\sO] \mid f$, until we find the largest one all of
whose elements are integral.  A much better algorithm is to proceed
exactly as just described, except use the ideas
of Section~\ref{sec:alg_pmax} to find a~$p$-maximal order for each prime
divisor of~$f$, then add these $p$-maximal orders together.

\begin{example}
  Consider the ring $\sO_K = \ZZ[(1+\sqrt{5})/2]$ of integers of
  $K=\QQ(\sqrt{5})$.  The discriminant of the basis $1,a=(1+\sqrt{5})/2$
  is
  \[
  \Disc(\sO_K) = \left| \begin{pmatrix} 2 & 1 \\ 1 & 3\end{pmatrix} \right| = 5.
  \]
  Let $\sO=\ZZ[\sqrt{5}]$ be the order generated by $\sqrt{5}$.
  Then~$\sO$ has basis $1,\sqrt{5}$, so
  \[
  \Disc(\sO) = \left| \begin{pmatrix} 2 & 0 \\ 0 & 10\end{pmatrix} \right| = 20 = [\sO_K:\sO]^2\cdot 5,
  \]
  hence~$[\sO_K:\sO] = 2$.
\end{example}

\begin{example}
  Consider the cubic field $K=\QQ(\sqrt[3]{2})$, and
  let $\sO$ be the order $\ZZ[\sqrt[3]{2}]$.
  Relative to the base $1,\sqrt[3]{2}, (\sqrt[3]{2})^2$ for~$\sO$,
  the matrix of the trace pairing is
  \[
    A = \begin{pmatrix}
     3 & 0 & 0 \\
     0 & 0 & 6 \\
     0 & 6 & 0
    \end{pmatrix}.
  \]
  Thus
  \[
    \Disc(\sO) = \det(A)= 108 = 2^2\cdot 3^3.
  \]
  Suppose we do not know that the ring of integers
  $\sO_K$ is equal to $\sO$.  By Proposition~\ref{prop:indsquare},
  we  have
  \[
    \Disc(\sO_K)\cdot [\sO_K:\sO]^2 = 2^2\cdot 3^3,
  \]
  so $3\mid \Disc(\sO_K)$, and $[\sO_K:\sO] \mid 6$.
  Thus to prove $\sO=\sO_K$ it suffices to prove
  that~$\sO$ is $2$-maximal and $3$-maximal,
  which could be accomplished as described in
  Section~\ref{sec:alg_pmax}.
\end{example}

\section{Norms of Ideals}
In this section we extend the notion of norm to ideals.  This will be
helpful in the next chapter, where
we will prove that the group of fractional ideals modulo principal
fractional ideals of a number field is finite by showing that every
ideal is equivalent to an ideal with norm at most some bound.
This is enough, because as we will see below there are only
finitely many ideals of bounded norm.
\begin{definition}[Lattice Index]
  If $L$ and $M$ are two lattices in a vector space $V$, then the
  \defn{lattice index} $[L:M]$ is by definition the absolute value of the
  determinant of any linear automorphism $A$ of $V$ such that $A(L)=M$.
\end{definition}
For example, if $L=2\ZZ$ and $M=10\ZZ$, then
\[
  [L:M] = [2\ZZ : 10\ZZ] = \det([5]) = 5,
\]
since $5$ multiplies $2\ZZ$ onto $10\ZZ$.

The lattice index has the
following properties:
\begin{itemize}
  \item If $M\subset L$, then $[L:M]=\#(L/M)$.
  \item If $M, L, N$ are any lattices in~$V$, then
  \[
    [L:N] = [L:M]\cdot [M:N].
  \]
\end{itemize}


\begin{definition}[Norm of Fractional Ideal]
  Suppose $I$ is a fractional ideal of $\sO_K$.  The \defn{norm} of~$I$ is
  the lattice index
  \[
    \Norm(I) = [\sO_K : I] \in \QQ_{\geq 0},
  \]
  or $0$ if $I=0$.
\end{definition}
Note that if $I$ is an integral ideal, then $\Norm(I)=\#(\sO_K/I)$.

\begin{lemma}\label{lem:aIfrac}\ilem{$\Norm(a I)$}
  Suppose $a \in K$ and $I$ is an integral ideal.
  Then
  \[
    \Norm(a I) = \left|\Norm_{K/\QQ}(a)\right| \Norm(I).
  \]
\end{lemma}
\begin{proof}
  By properties of the lattice index mentioned above we have
  \[
    [\sO_K : aI] = [\sO_K : I] \cdot [I:aI]
    = \Norm(I) \cdot \left|\Norm_{K/\QQ}(a)\right|.
  \]
  Here we have used that $[I:aI]=\left|\Norm_{K/\QQ}(a)\right|$, which is because left
  multiplication $\ell_a$ by $a$ is an automorphism of $K$ that sends $I$ onto
  $aI$, so
  \[
    [I:aI]=\left|\det(\ell_a)\right|=\left|\Norm_{K/\QQ}(a)\right|.
  \]
\end{proof}

\begin{proposition}\iprop{multiplicativity of ideal norm}
  If $I$ and $J$ are fractional ideals, then
  \[
    \Norm(IJ) = \Norm(I)\cdot \Norm(J).
  \]
\end{proposition}
\begin{proof}
  By Lemma~\ref{lem:aIfrac}, it suffices to prove this when $I$ and $J$ are
  integral ideals.  If $I$ and $J$ are coprime, then
  Theorem~\ref{thm:crt} (the Chinese Remainder Theorem) implies that
  $\Norm(IJ) = \Norm(I)\cdot \Norm(J)$.  Thus we reduce to the case when
  $I=\p^m$ and $J=\p^k$ for some prime ideal $\p$ and integers $m,k$.
  By Proposition~\ref{prop:quopow}, which is
  a consequence of CRT, the filtration of $\sO_K/\p^{n}$ given
  by powers of~$\p$ has successive quotients isomorphic to $\sO_K/\p$.
  \todo{Write more, right after I teach this.}
  Thus  we see that $\#(\sO_K/\p^{n}) = \#(\sO_K/\p)^{n}$, which proves that
  $\Norm(\p^n)=\Norm(\p)^n$.
\end{proof}

\begin{example}
  We compute some ideal norms using {\Sage}.
\begin{sagecode}
\begin{sagecell}
K.<a> = NumberField(x^2 - 5)
I = K.fractional_ideal(a)
I.norm()
\end{sagecell}
\begin{sageout}
5
\end{sageout}
\begin{sagecell}
J = K.fractional_ideal(17)
J.norm()
\end{sagecell}
\begin{sageout}
289
\end{sageout}
\end{sagecode} %link
\noindent{} We can also use functional notation:
\begin{sagecode} %link
\begin{sagecell}
norm(I*J)
\end{sagecell}
\begin{sageout}
1445
\end{sageout}
\end{sagecode}
\end{example}

We will use the following proposition in the next chapter when
we prove finiteness of class groups.
\begin{proposition}\label{prop:finitewithnorm}%
  \ilem{integral ideals of bounded norm}
  Fix a number field $K$.
  Let $B$ be a positive integer.  There
  are only finitely many integral ideals
  $I$ of $\sO_K$ with norm at most $B$.
\end{proposition}
\begin{proof}
  An integral ideal $I$ is a subgroup of $\sO_K$ of index equal to the
  norm of $I$.  If $G$ is any finitely generated abelian group, then
  there are only finitely many subgroups of $G$ of index at most $B$.
  This is because the subgroups of index dividing an integer $n$ are all subgroups
  of $G$ that contain $nG$, and the group $G/nG$ is finite.
\end{proof}
