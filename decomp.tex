\chapter{Decomposition and Inertia Groups}

In this chapter we will study extra structure in the case when~$K$
is Galois over~$\Q$. We will learn about Frobenius elements,
the Artin symbol, decomposition groups, and how the Galois group of
$K$ is related to Galois groups of residue class fields. These are
the basic structures needed to attach $L$-functions to representations of
$\galq$, which will play a central role in the next few chapters.

\section{Galois Extensions}

In this section we give a survey (no proofs) of the basic facts about
Galois extensions that will be needed in the rest of this chapter.
\begin{definition}[Galois]
	An extension $L/K$ of number fields is \defn{Galois} if
	$$
		\#\Aut(L/K) = [L:K],
	$$
	where $\Aut(L/K)$ is the group of automorphisms of $K$
	that fix $L$.  We write
	$$
		\Gal(L/K) = \Aut(L/K).
	$$
\end{definition}
For example, if $K\subset \C$ is a number field embedded in the complex numbers,
then $K$ is \defn{Galois} over $\Q$ if every field
homomorphism $K\to \C$ has image $K$. As another example, any quadratic
extension $L/K$ is Galois over $K$, since it is of the form $K(\sqrt{a})$,
for some $a\in K$, and the nontrivial automorphism is induced
by $\sqrt{a}\mapsto -\sqrt{a}$, so there is always one nontrivial automorphism.
If $f\in K[x]$ is an irreducible cubic polynomial, and $\alpha$ is
a root of $f$, then one proves in a course on Galois theory that $K(\alpha)$
is Galois over $K$ if and only if the discriminant of~$f$ is a perfect square
in~$K$. ``Random'' number fields of degree bigger than $2$ are rarely Galois.

If $K\subset \C$ is a number field, then the \defn{Galois closure} $K^{\gc}$
of $K$ in $\C$ is the field generated by all images of~$K$ under all
embeddings in~$\C$ (more generally, if $L/K$ is an extension, the
Galois closure of $L$ over $K$ is the field generated by images of
embeddings $L\to \C$ that are the identity map on $K$).

\begin{exercise}
	Suppose $K \subset \C$ is a number field of the form $\Q(\alpha)$ for some
	$\alpha \in \C$. Show that $K^{\gc}$ is generated (as an extension of $\Q$)
	by all the conjugates of $\alpha$.
%the image under an embedding of any
%polynomial in the conjugates of~$\alpha$ is again a polynomial in
%conjugates of $\alpha$.
\end{exercise}

How much bigger can the degree of $K^{\gc}$ be as compared to the
degree of $K=\Q(\alpha)$? There is an embedding of $\Gal(K^{\gc}/\Q)$
into the group of permutations of the conjugates of $\alpha$.
If $\alpha$ has $n$ conjugates, then this is an embedding
$\Gal(K^{\gc}/\Q)\hra S_n$, where $S_n$ is the symmetric group on~$n$
symbols, which has order~$n!$. Thus the degree of the $K^{\gc}$ over~$\Q$
is a divisor of $n!$. Also $\Gal(K^{\gc}/\Q)$ is a transitive
subgroup of $S_n$, which constrains the possibilities further.
When $n=2$, we recover the fact that quadratic extensions are Galois.
When $n=3$, we see that the Galois closure of a cubic extension is either
the cubic extension or a quadratic extension of the cubic extension.
One can show that the Galois closure of a cubic extension is obtained
by adjoining the square root of the discriminant, which is why an
irreducible cubic defines a Galois extension if and only if the discriminant
is a perfect square.

For an extension $K$ of $\Q$ of degree $5$, it is ``frequently'' the case that
the Galois closure has degree $120$, and in fact it is an interesting problem
to enumerate examples of degree~$5$ extensions in which the Galois closure
has degree smaller than $120$. For example, the only possibilities for the
order of a transitive proper subgroup of $S_5$ are $5$, $10$, $20$, and $60$;
there are also proper subgroups of $S_5$ order $2, 3, 4, 6, 8, 12$, and $24$,
but none are transitive.

\begin{exercise}
	Let $\alpha$ be a root of the irreducible polynomial $f(x) = x^5 - 6x + 3$ and
	let $K = \Q(\alpha)$.
	
	\begin{enumerate}
		\item
		Use \sage to verify that the Galois group $\Gal(K^{\gc}/\Q)$
		has order $120$.
		Warning: this command may take a long time to run.
		Try to finish the second part of this exercise before your code
		finishes.
%%%time
%R.<x> = ZZ[]
%f = x^5 - 6*x + 3
%K.<a> = NumberField(f)
%Kgc.<b> = K.galois_closure()
		
		\item
		One can show that $f$ has three real roots and two complex roots.
		Show that $\Gal(K^{\gc}/\Q)$ contains an element of order $5$ and
		an element of order $5$. Use this to argue that $\Gal(K^{\gc})/\Q$
		has order $120$.
	\end{enumerate}
	
	\begin{hint}
		Number fields in \sage have a {\tt galois\_closure()} command that
		returns the Galois closure of the field.
		For the second part, you want to show that any $5$-cycle and transposition
		will generate $S_5$.
	\end{hint}
\end{exercise}

\begin{example}
	Let $n$ be a positive integer. Consider the field $K=\Q(\zeta_n)$,
	where $\zeta_n=e^{2\pi i/n}$ is a primitive $n$th root of unity.  If
	$\sigma:K\to \C$ is an embedding, then $\sigma(\zeta_n)$ is also an
	$n$th root of unity, and the group of $n$th roots of unity is cyclic.
	So $\sigma(\zeta_n) = \zeta_n^m$ for some $m$ which is invertible
	modulo $n$.  Thus $K$ is Galois and $\Gal(K/\Q)\hra (\Z/n\Z)^*$.
	However, $[K:\Q]=\vphi(n)$, so this map is an isomorphism.
\end{example}

\begin{remark}
	Taking a limit using the maps $\Gal(\Qbar/\Q)\to
	\Gal(\Q(\zeta_{p^r})/\Q)$, we obtain a homomorphism $\Gal(\Qbar/\Q)\to
	\Z_p^*$, which is called the {\em $p$-adic cyclotomic character}.
\end{remark}

Compositums of Galois extensions are Galois.  For example, the
biquadratic field $K=\Q(\sqrt{5},\sqrt{-1})$ is a Galois
extension of $\Q$ of degree~$4$, which is the compositum
of the Galois extensions $\Q(\sqrt{5})$ and $\Q(\sqrt{-1})$ of $\Q$.

Fix a number field $K$ that is Galois over $\Q$.
Then the Galois group acts on many of the objects
that we have associated to $K$.

\begin{exercise}\label{ex:galois-action-on-things}
	Let $L/K$ be a Galois extension of number fields, and let $G = \Gal(L/K)$.
	Describe the natural action of $G$ on the following objects:
	\begin{itemize}
		\item The ring of integers $\O_K$.
		\item The group units $U_K$.
		\item The set of ideals of $\O_K$.
		\item The group of fractional ideals of $\O_K$.
		\item The class group $\Cl(K)$.
		\item The set $S_\p$ of prime ideals of $\O_L$ lying over a given nonzero
		prime ideal $\p$ of $\O_K$, i.e., the prime divisors of $\p\O_L$.
	\end{itemize}
\end{exercise}

In the next section we will be concerned with the action of
$\Gal(L/K)$ on $S_\p$, though actions on each of the other objects,
especially $\Cl(L)$, are also of great interest. Understanding the
action of $\Gal(L/K)$ on $S_{\p}$ will enable us to associate, in a
natural way, a holomorphic $L$-function to any complex representation
$\Gal(L/K) \to \GL_n(\C)$.

\section{Decomposition of Primes: $efg=n$}

Let $L/K$ be an extension of number fields and let $\p$ be a prime
in $\O_K$. By Theorem~\ref{thm:intuniqfac} the ideal $\p\O_L$ factors
uniquely into a product of primes of $\O_L$ given by
$$
	\p\O_L = \prod_{i=1}^g \q^{e_i}_i,
$$
where the $\q_i$ are the prime ideals of $\O_L$ laying over $\p$, and
the $e_i$ are positive integers. The goal of this section is to study this
factorization. First we will introduce some standard terminology.

\begin{definition}[Ramification Index]
	The \defn{ramification index} of $\q_i$ over $\p$ is
	$$
		e(\q_i/\p) = e_i.
	$$
\end{definition}

\begin{definition}[Inertia degree]
	The \defn{inertia degree}
	of $\q_i$ over $\p$ is
	$$
		f(\q_i/\p) = \left[\O_L/\q_i : \O_K/\p\right].
	$$
\end{definition}

\begin{exercise}\label{ex:ramificationmultiplicative}
	The following properties follow quickly from the definitions.
	Let $M/L/K$ be a tower of number fields. Let $\p$ be a prime
	in $\O_K$, $\q$ a prime in $\O_L$ lying over $\p$, and $\P$ a prime
	in $\O_M$ lying over $\q$.
	\begin{enumerate}
		\item[(a)] Show that $e(\P/\p)=e(\P/\q) \cdot e(\q/\p)$.
		
		\item[(b)] Show that $f(\p/p)=f(\P/\q) \cdot f(\q/\p)$.
		
		\item[(c)] Let $g_{L/K}(\p)$ be the number of primes
		of $\O_L$ lying over $\p$. Show that
		$$
			g_{M/K}(\p) = \sum\limits_{\q \text{ divides } \p\O_L} g_{L/K}(\q).
		$$
	\end{enumerate}
\end{exercise}


Now suppose $L/K$ is Galois and let $\sigma\in\Gal(L/K)$.
We saw in Exercise~\ref{ex:galois-action-on-things} that $\Gal(L/K)$ acts
naturally on the set $S_\p$ for a prime $\p$ of $\O_K$.
This means that $\sigma(\q) \in S_\p$ for any $\q \in S_\p$. Moreover,
$\sigma$ induces an isomorphism of finite fields $\O_L/\q\to \O_L/\sigma(\q)$
that fixes the common subfield $\O_K/\p$. Thus $\q$ and $\sigma(\q)$ have
the same inertia degree, i.e. $f(\q/\p) = f(\sigma(\q)/\p)$.
In fact, much more is true.




\begin{theorem}\label{thm:transitive}\ithm{transitive Galois action}
	Suppose $L/K$ is a Galois extension of number fields,
	and let $\p$ be a prime of $\O_K$. 
	Write $\p\O_K = \prod_{i=1}^g \q_i^{e_i}$, and let $f_i = f(\q_i/\p)$.
	Then $G = \Gal(L/K)$ acts transitively on the set $S_\p$ of primes $\q_i$,
	and
	$$
		e_1 = \cdots = e_g, \qquad f_1 = \cdots = f_g.
	$$
	Moreover, if we let $e$ be the common value of the $e_i$,
	$f$ the common value of the $f_i$, and $n = [K:L]$, then
	$$
		efg = n.
	$$
\end{theorem}
\begin{proof}
	For simplicity, we will give the proof only for an extension $K/\Q$, but
	the proof works in general. Suppose $p\in\Z$ and
	$p\O_K=\p_1^{e_1}\cdots \p_g^{e_g}$, and $S=\{\p_1,\ldots, \p_g\}$.  We
	will first prove that $G = \Gal(K/\Q)$ acts transitively on $S$. 
	Let $\p=\p_i$ for some~$i$.
	Recall Lemma~\ref{lem:magica} which we proved long ago using the
	Chinese Remainder Theorem (Theorem~\ref{thm:crt}). It showed there exists
	$a\in\p$ such that $(a)/\p$ is an integral ideal that is
	coprime to $p\O_K$.   The product
	\begin{equation}\label{eqn:prodquo}
		I= \prod_{\sigma\in G} \sigma((a)/\p)
		= \prod_{\sigma\in G} \frac{(\sigma(a))\O_K}{\sigma(\p)}
		= \frac{(\Norm_{K/\Q}(a))\O_K}{\displaystyle \prod_{\sigma\in G} \sigma(\p)}
	\end{equation}
	is a nonzero integral $\O_K$ ideal since it is a product of nonzero
	integral $\O_K$ ideals.
	Since $a\in \p$ we have that
	$\Norm_{K/\Q}(a) \in \p\cap\Z=p\Z$.  Thus the numerator of
	the rightmost expression in (\ref{eqn:prodquo}) is
	divisible by $p\O_K$.   Also, because $(a)/\p$ is coprime
	to $p\O_K$, each $\sigma((a)/\p)$ is coprime to $p\O_K$
	as well.   Thus $I$ is coprime to $p\O_K$.   This means the
	denominator of the rightmost expression in (\ref{eqn:prodquo})
	must also be divisible by $p\O_K$ in order to cancel the $p\O_K$
	in the numerator.  Thus we have shown that for any~$i$,
	$$
		\prod_{j=1}^g \p_j^{e_j}
		= p\O_K \,\,\Big|\,\, \prod_{\sigma\in G} \sigma(\p_i).
	$$
	By unique factorization, since every $\p_j$ appears in the left hand
	side, we must have that for each~$j$ there is a~$\sigma$ with
	$\sigma(\p_i)=\p_j$, i.e., $G$ acts transitively on $S$.

	Choose some $j$ and suppose that $k\neq j$ is another index.  Because
	$G$ acts transitively, there exists $\sigma\in G$ such that
	$\sigma(\p_k)=\p_j$.  Applying $\sigma$ to the factorization $p\O_K =
	\prod_{i=1}^g \p_i^{e_i}$, we see that
	$$
		\prod_{i=1}^g \p_i^{e_i} = \prod_{i=1}^g \sigma(\p_i)^{e_i}.
	$$
	Using unique factorization,
	we get $e_j = e_k$.  Thus $e_1=e_2=\cdots = e_g$.

	As was mentioned right before the statement of the theorem,
	for any $\sigma\in G$ we have $\O_K/\p_i\isom \O_K/\sigma(\p_i)$.
	Since $G$ acts transitively it follows that $f_1=f_2=\cdots = f_g$.
	We have, upon applying the Chinese Remainder Theorem
	and noting $\#(\O_K/(\p^m)) = \#(\O_K/\p)^m$
	(see Exercise~\ref{ex:residuefieldofpower}), that
	\begin{align*}
		[K:\Q]&= \dim_{\Z} \O_K = \dim_{\F_p} \O_K/p\O_K \\
		&= \dim_{\F_p} \left(\bigoplus_{i=1}^g \O_K/\p_i^{e_i}\right)
		= \sum_{i=1}^g e_i f_i
		= efg,
	\end{align*}
	which completes the proof.
\end{proof}

\subsection{Examples}

This section gives examples illustrating the theorem for quadratic fields
and a cubic field and its Galois closure.

\subsubsection{Quadratic Extensions}

Suppose $K/\Q$ is a quadratic field. 
Then $K$ is Galois, so for each prime $p\in\Z$ we have $2 = efg$.
There are exactly three possibilities for $e$, $f$ and $g$ :
\begin{description}
	\item[\normalfont{(Ramified):}] $e=2$, $f=g=1$: The prime $p$ \emph{ramifies} in
	$\O_K$, which means $p\O_K = \p^2$.  Let $\alpha$ be a generator for $\O_K$ and
	$h\in\Z[x]$ a minimal polynomial for $\alpha$.
	By Theorem~\ref{thm:fac1} a prime $p$ is ramified in $\O_K$ if and only if
	$h$ has a double root modulo~$p$, which is equivalent to $p$ dividing
	the discriminant of $h$. This shows there are only finitely many ramified
	primes.

	\item[\normalfont{(Inert):}] $e=1$, $f=2$, $g=1$: The prime $p$ is \emph{inert} in $\O_K$,
	which means $p\O_K = \p$ is prime.  It is a nontrivial theorem that
	this happens half of the time,
	as we will see illustrated below for a particular example.

	\item[\normalfont{(Split):}] $e=f=1$, $g=2$: The prime $p$ \emph{splits} in $\O_K$,
	which means $p\O_K = \p_1\p_2$ with $\p_1\neq \p_2$.  This happens the other
	half of the time.
\end{description}

\begin{example}\label{exam:decompQsqrt5}
	Let $K=\Q(\sqrt{5})$, so $\O_K=\Z[\gamma]$, where
	$\gamma=(1+\sqrt{5})/2$.  Then $p=5$ is ramified, since $5\O_K =
	(\sqrt{5})^2$.  More generally, the order $\Z[\sqrt{5}]$ has index $2$
	in $\O_K$, so for any prime $p\neq 2$ we can determine the
	factorization of $p$ in $\O_K$ by finding the factorization of the
	polynomial $x^2-5\in \F_p[x]$.  The polynomial $x^2-5$ splits as a
	product of two distinct factors in $\F_p[x]$ if and only if $e=f=1$
	and $g=2$.  For $p\neq 2,5$ this is the case if and only if $5$ is a
	square in $\F_p$, i.e., if $\kr{5}{p} = 1$, where $\kr{5}{p}$ is $+1$
	if $5$ is a square mod $p$ and $-1$ if $5$ is not.  By quadratic
	reciprocity,
	$$
		\kr{5}{p}
		= (-1)^{\frac{5-1}{2}\cdot \frac{p-1}{2}} \cdot \kr{p}{5}
		= \kr{p}{5}
		= \begin{cases}
			+1 & \text{ if } p\con \pm 1\pmod{5} \\
			-1 & \text{ if } p \con \pm 2\pmod{5}.
		\end{cases}
	$$
	Thus whether $p$ splits or is inert in
	$\O_K$ is determined by the residue class of~$p$
	modulo $5$. It is a theorem of Dirichlet, which was massively
	generalized by Chebotarev, that $p\con \pm 1$ half the time
	and $p \con \pm 2$ the other half the time.\footnote{
	For the actual statement and a proof of this theorem,
	see \cite{neukirch1999} Theorem~VII.13.4.}
\end{example}

\subsubsection{The Cube Root of Two}

Suppose $K/\Q$ is not Galois.
Then $e_i$, $f_i$, and~$g$ are defined for each prime $p\in\Z$,
but we need not have $e_1=\cdots=e_g$ or $f_1=\cdots =f_g$.  We do still have that
$\sum_{i=1}^g e_i f_i = n$, by the Chinese Remainder Theorem as used in
the proof of \autoref{thm:transitive}.
%also \cite[Thm.~21]{marcus1977number}
%or \cite[Prop.~I.8.2]{neukirch1999}

Consider the case where $K=\Q(\sqrt[3]{2})$. We know that $\O_K = \Z[\sqrt[3]{2}]$.  Thus
$2\O_K = (\sqrt[3]{2})^3$, so for $2$ we have $e=3$ and $f=g=1$.

Working modulo $5$ we have
$$
	x^3 - 2 = (x+2)(x^2+3x+4) \in \F_5[x],
$$
and the quadratic factor is irreducible.  Thus
$$
	5\O_K = \left(5, \crtwo+2\right) \cdot \left(5, \left(\crtwo\right)^2 + 3\crtwo + 4\right).
$$
Thus here $g = 2$, $e_1 = e_2 = 1$, $f_1=1$, and $f_2 = 2$.
Thus when $K$ is not Galois we need not have that the $f_i$
are all equal.

\subsection{Definitions and Terminology}

In the previous sections we used words like ``ramify'',
``inert'', and ``split'' to describe the decomposition
of a prime in an extension. This section will define these
terms which will be used in later sections.

Let $L/K$ be an extension of number fields of degree $n$,
and let $\p$ be a prime in $\O_K$. We have the usual factorization
$$
	\p\O_L = \prod_{i=1}^g \q_i^{e_i}
$$
where the $\q_i$ ranger over the primes of $\O_L$ laying over $\p$.

\begin{definition}\label{def:ramify}
	The prime $\p$ \emph{ramifies} in $L$ if $e_i > 1$ for some $1\leq i\leq g$.
	Otherwise $\p$ is \emph{unramified}.
	%in general, residue fields should be separable, neukrich 49
	If also $g = 1$ and $f_1 = 1$, then $\p$ is \emph{totally ramified}.
\end{definition}

\begin{definition}\label{def:inert}
	The prime $\p$ is \emph{inert} in $L$ if $\p\O_L$ is prime.
	In this case we have $g = 1$ and $e_1 = 1$.
\end{definition}

\begin{definition}\label{def:split}
	The prime $\p$ \emph{splits} in $L$ if $g > 1$. If also $g = [L : K]$,
	then $\p$ \emph{splits completely} or is \emph{totally split}.
\end{definition}

\begin{exercise}[See {\cite[Ch.~4, Exercise~24]{marcus1977number}}]
	Prove the following properties.
	\begin{enumerate}
		\item[(a)] If $\p$ it totally ramified in $L$
		then it is totally ramified in $K$.
		
		\item[(b)] Let $L'$ be another extension of $K$.
		If $\p$ is totally ramified in $L$ and unramified in $L'$
		then $L\cap L' = K$.
	\end{enumerate}
\end{exercise}

\begin{exercise}
	Let $K$ be a number field and $d_K$ the discriminant of $K$.
	Prove that $p\mid d_K$ if and only if $p$ ramifies in $K$.

	\begin{hint}
		This is proved in many books, see for example
		\cite[Thm.~24]{marcus1977number} or \cite[Cor.~III.2.12]{neukirch1999}
	\end{hint}
\end{exercise}

\section{The Decomposition Group}

Suppose $K$ is a number field that is Galois over $\Q$ with
group $G=\Gal(K/\Q)$. Fix a prime $\p\subset \O_K$ lying over $p\in\Z$.
\begin{definition}[Decomposition group]\label{def:decomp}
	The \defn{decomposition group} of $\p$ is the subgroup
	$$
		D_\p = \{\sigma \in G : \sigma(\p)=\p\} \subset G.
	$$
\end{definition}
Note that $D_\p$ is the stabilizer of $\p$ for
the action of $G$ on the set of primes lying over $p$.

It also makes sense to define decomposition groups for relative
extensions $L/K$, but for simplicity and to fix ideas in this section
we only define decomposition groups for a Galois extension $K/\Q$.

Let $\F_{\p} = \O_K/\p$ denote the residue class field of $\p$.
In this section we will prove that there is an exact sequence
$$
	1\to I_\p \to D_\p \to \Gal(\F_{\p}/\F_p)\to 1,
$$
where $I_\p$ is the \defn{inertia subgroup} of $D_\p$, and $\#I_\p=e = e(\p/p)$.
The most interesting part of the proof is showing that the natural
map $D_\p\to  \Gal(\F_{\p}/\F_p)$ is surjective.
We will also discuss the structure of $D_\p$ and introduce
Frobenius elements, which play a crucial role in understanding Galois
representations.


Recall from Theorem~\ref{thm:transitive} that~$G$ acts transitively
on the set of primes~$\p$ lying over~$p$.
The orbit-stabilizer theorem implies that $[G:D_\p]$ equals the
cardinality of the orbit of~$\p$, which by Theorem~\ref{thm:transitive}
equals the number~$g$ of primes lying over~$p$, so $[G:D_\p]=g$.

\begin{lemma}\label{decompGpsConj}\ilem{decomposition groups are conjugate}
	The decomposition subgroups $D_\p$ corresponding to primes $\p$
	lying over a given $p$ are all conjugate as subgroups of~$G$.
\end{lemma}
\begin{proof}
	See Exercise~\ref{ex:decompGpsConj}.
\end{proof}

\begin{exercise}\label{ex:decompGpsConj}
	Prove Lemma~\ref{decompGpsConj}.

	\begin{hint}
		For $\sigma,\tau\in G$ you need to show
		$\tau D_\p \tau^{-1} = D_{\tau\p}$.
		Start by writing down what it means for $\sigma\in D_\p$
		and $\tau\sigma\tau^{-1}\in D_{\tau\p}$.
	\end{hint}
%solution:
%We have for each $\sigma, \tau \in G$, that
%$$\tau^{-1}\sigma \tau\p = \p
%\iff
%\sigma\tau \p = \tau \p,
%$$
%so
%$$
%\sigma \in D_{\tau\p} \iff \tau^{-1}\sigma\tau\in D_\p.
%$$
%Thus
%$$
% \sigma \in D_\p \iff \tau \sigma \tau^{-1} \in D_{\tau \p},
%$$
%which shows $\tau D_{\p}\tau^{-1} = D_{\tau \p}$.
\end{exercise}

The decomposition group is useful because it allows us
to refine the extension $K/\Q$ into a tower of extensions, such that at
each step in the tower we understand the splitting behavior
of the primes lying over~$p$.

Recall the correspondence between subgroups of the Galois group
$G$ and subfields of $K$. The fixed fields corresponding to the
decomposition and inertia subgroups have an important description
in terms of the splitting behavior of the prime $\p$.
We characterize the fixed field of $D=D_\p$ as follows.

\begin{proposition}\label{prop:nosplit}\iprop{fixed field characterization}
	The fixed field
	$$
		K^D = \left\{a \in K \colon \sigma(a) = a\text{ for all }\sigma \in D\right\}
	$$
	of $D$ is the smallest subfield $F \subset K$ such that
	there is a unique prime of $\O_K$ lying over $\q = \p\cap\O_F$.
\end{proposition}
\begin{proof}
	First suppose $F = K^D$, and note that by Galois theory
	$\Gal(K/F)\isom D$. By Theorem~\ref{thm:transitive}, the group $D$
	acts transitively on the primes of $K$ lying over $\q$.  One of
	these primes is $\p$, and $D$ fixes $\p$ by definition, so there is
	only one prime of $K$ lying over $\q$.
	Conversely, if $F\subset K$ is such that $\q$ lies under a unique prime
	in $K$, then $\Gal(K/F)$ fixes $\p$ (since it is the only
	prime over $\q$), so $\Gal(K/F) \subset D$, hence $K^D \subset F$.
\end{proof}

Thus $p$ does not split in going from $K^D$ to $K$---it does some
combination of ramifying and staying inert.
To fill in more of the picture, the following proposition asserts that $p$
splits completely and does not ramify in $K^D/\Q$.

\begin{proposition}\label{prop:noresidue}\iprop{$e$, $f$, $g$}
	Fix a finite Galois extension~$K$ of~$\Q$,
	let~$\p$ be a prime lying over~$p$ with decomposition group~$D$,
	and set $F = K^D$ and $\q = \p \cap \O_F$.
	Let $g$ be the number of primes of $K$ lying over $p$.
	Then
	$$
		e(\q/p) = f(\q/p) = 1,
		\quad e(\p/p) = e(\p/\q)
		\quad f(\p/p) = f(\p/\q),
		\quad \text{and } g = [F : \Q].
	$$
\end{proposition}
\begin{proof}
	As mentioned right after Definition~\ref{def:decomp}, the
	orbit-stabilizer theorem implies that $g = [G : D]$, and
	by Galois theory $[G : D] = [F : \Q]$, so $g = [F : \Q]$. By
	Proposition~\ref{prop:nosplit}, $\p$ is the only prime of $K$
	lying over $\q$ so by Theorem~\ref{thm:transitive},
	\begin{align*}
		e(\p/\q) \cdot f(\p/\q) = [K:F]
		&= \frac{[K:\Q]}{[F:\Q]} \\
		&= \frac{e(\p/p) \cdot f(\p/p) \cdot g}{[F:\Q]} \\
		&= e(\p/p) \cdot f(\p/p).
	\end{align*}
	Now $e(\p/\q)\leq e(\p/p)$ and $f(\p/\q)\leq f(\p/p)$, so
	we must have $e(\p/\q)=e(\p/p)$ and $f(\p/\q)=f(\p/p)$.
	Since from Exercise~\ref{ex:ramificationmultiplicative} we have
	$e(\p/p) = e(\p/\q) \cdot e(\q/p)$ and $f(\p/q) = f(\p/\q) \cdot f(\q/p)$,
	it follows that $e(\q/p) = f(\q/p) = 1$.
\end{proof}

We summarize the results of the decomposition of a prime in the
tower $K \supseteq K^D \supseteq \Q$ in Table~\ref{tbl:decompfield}.
This table shows the ramification indices, inertia degrees,
and the number of primes at each step of the tower.

\begin{table}[h!]
	\centering
	\begin{tabular}{ >{$}c<{$} >{$}c<{$} >{$}c<{$} | >{$}c<{$} >{$}c<{$} }
		\text{Ramification ($e$)} & \text{Inertia ($f$)} & \text{Splitting ($g$)} & \text{Primes} & \text{Fields} \\
		\hline
		 &  &  & \p & K \\
		e(\p/p) & f(\p/p) & 1 & \vert & \vert \\
		 &  &  & \q & L \\
		1 & 1 & [K^D : \Q] & \vert & \vert \\
		 &  &  & p &  \Q
	\end{tabular}
	\caption{Decomposition in the fixed field $K^D$.}
	\label{tbl:decompfield}
\end{table}

\begin{exercise}
	Give an example of each of the following:
	\begin{enumerate}
		\item A finite nontrivial Galois extension $K$
		of $\Q$ and a prime ideal $\p$ such that $D_\p = \Gal(K/\Q)$.
		\item A finite nontrivial Galois extension $K$ of
		$\Q$ and a prime ideal $\p$ such that $D_\p$ has order~$1$.
		\item A finite Galois extension~$K$ of
		$\Q$ and a prime ideal $\p$ such that $D_\p$ is not a normal
		subgroup of $\Gal(K/\Q)$.
		\item A finite Galois extension $K$ of
		$\Q$ and a prime ideal $\p$ such that $I_\p$ is not a normal
		subgroup of $\Gal(K/\Q)$.
	\end{enumerate}
\end{exercise}

\subsection{Galois groups of finite fields}\label{sec:galoisfinite}

Each $\sigma \in D = D_\p$ acts in a well-defined
way on the finite field $\F_\p = \O_K/\p$, so we obtain
a homomorphism
$$
	\vphi:D_\p \to \Aut(\F_\p/\F_p).
$$
We pause for a moment and review a few basic properties of
extensions of finite fields. In particular, they turn out
to be Galois so the map $\vphi$ above is actually a map
$D_\p \to \Gal(\F_\p/\F_p)$.
The properties in this section are general properties
of Galois groups for finite fields.

\begin{definition}
	Let $k$ be any field of characteristic $p$.
	Define $\Frob_p:k\to k$ to be the homomorphism
	given by $a\mapsto a^p$. The map $\Frob_p$ is
	called the \emph{Frobenius} homomorphism.
\end{definition}

\begin{exercise}\label{ex:frob}
	\hfill
	\begin{enumerate}
		\item
		Show the map $\Frob_p$ is in fact a field homomorphism,
		that is $\Frob_p(a + b) = \Frob_p(a) + \Frob_p(b)$
		and $\Frob_p(ab) = \Frob_p(a)\Frob_p(b)$.

		\item
		Suppose $k = \F_p$. Then show $\Frob_p = id$, i.e.,
		$a^p = a$ for any $a\in\F_p$.

		\item
		Suppose $k = \F_{q}$ where $q=p^f$ for some $f\geq 1$.
		Show that $\Frob_p:k\to k$ is an automorphism.

		\item
		Continuing the previous part, note that by
		Exercise~\ref{ex:finitesubgroupoffieldcyclic}, $k^*$ is cyclic.
		Let $a\in k$ be a generator for $k^*$,
		so $a$ has multiplicative order $p^f-1$ and $k = \F_p(a)$.
		Show that
		$$
			\Frob_p^n(a) = a^{p^n} = a
			\quad\Leftrightarrow\quad (p^f - 1) \mid p^n - 1
			\quad\Leftrightarrow\quad f \mid n
		$$
	\end{enumerate}
\end{exercise}

\begin{remark}
	Exercise~\ref{ex:frob} shows that all finite fields
	are \emph{perfect}. For more on perfect fields see
	a standard abstract algebra text such as
	\cite{dummit2004abstract}.
\end{remark}

By Exercise~\ref{ex:frob}(b,c) the map $\Frob_p$ is an
automorphism of $\F_\p$ fixing $\F_p$ and hence defines
an element in $\Gal(\F_\p/\F_p)$. Let $f = f_{\p/p}$ be the residue
degree of $\p$, i.e., $f = [\F_\p:\F_p]$.
Exercise~\ref{ex:frob}(d) shows the order of $\Frob_p$ is~$f$.
Since the order of the automorphism group of a field extension
is at most the degree of the extension, we conclude that
$\Aut(\F_\p/\F_p)$ is generated by $\Frob_p$. This shows
$\Aut(\F_\p/\F_p)$ has order equal to the degree $[\F_\p/\F_p]$
so we conclude that $\F_\p/\F_p$ is Galois.
We summarize the discussion into the following theorem.

\begin{theorem}\label{thm:galoisgroupfinitefield}
	The extension $\F_{\p}/\F_p$ is Galois and moreover,
	$\Gal(\F_{\p}/\F_p)$ is generated by the Frobenius map
	$\Frob_p$ defined by $a\mapsto a^p$.
\end{theorem}

\begin{exercise}
	Prove that up to isomorphism there is
	exactly one finite field of each degree.

	\begin{hint}
		By Theorem~\ref{thm:galoisgroupfinitefield}
		all elements in a finite field satisfy an equation
		of the form $x^{p^f} - x$ where $p$ is the
		characteristic and $f$ is the degree over $\F_p$.
	\end{hint}
\end{exercise}


\subsection{The Exact Sequence}\label{sec:exactseq}

Because $D_\p$ preserves $\p$, there is a natural reduction homomorphism
$$
	\vphi:D_\p \to \Gal(\F_{\p}/\F_p).
$$
\begin{theorem}\label{thm:redsurj}\ithm{reduction of Galois group}
	The homomorphism $\vphi$ is surjective.
\end{theorem}
\begin{proof}
	Let $D = D_\p$ and $\tilde{a} \in \F_{\p}$ be an element
	such that $\F_{\p} = \F_p(\tilde{a})$.
	Lift $\tilde{a}$ to an algebraic integer $a\in \O_K$, and let
	$h = \prod_{\sigma \in D}(x-\sigma(a))\in K^D[x]$.
	Let $\tilde{h}$ be the reduction of $h$ modulo~$\p$.
	Note that $h(a) = 0$ so $\tilde{h}(\tilde{a}) = 0$.
	
	Note that the coefficients of $h$ lie in $\O_{K^D}$.
	By Proposition~\ref{prop:noresidue}, the residue field of $\O_{K^D}$
	is $\F_p$ so $\tilde{h}\in\F_p[x]$.
	Therefore $\tilde{h}$ is a multiple of the minimal polynomial of
	$\tilde{a}$ over $\F_p$. In particular, $\Frob_p(\tilde{a})$
	must also be a root of $\tilde{h}$.
	Since the roots of $\tilde{h}$ are of the form $\widetilde{\sigma(a)}$
	this shows that $\widetilde{\sigma(a)} = \Frob_p(\tilde{a})$
	for some $\sigma\in D$.
	Hence $\varphi(\sigma)(\tilde{a}) = \Frob_p(\tilde{a})$. Since elements
	of $\Gal(K_\p/\F_p)$ are determined by their action on $\tilde{a}$
	by choice of $\tilde{a}$, it follows that $\vphi(\sigma) = \Frob_p$
	and hence $\varphi$ is surjective because $\Frob_p$
	generates $\Gal(\F_{\p}/\F_p)$.
\end{proof}

\begin{definition}[Inertia Group]
	The \defn{inertia group associated to $\p$}
	is the kernel $I_\p$ of the map $D_\p\to\Gal(\F_{\p}/\F_p)$.
\end{definition}
We have an exact sequence of groups
\begin{equation}\label{eqn:exact}
	1 \to I_\p \to D_\p \to \Gal(\F_{\p}/\F_p)\to 1.
\end{equation}
The inertia group is a measure of how $p$ ramifies in $K$.
\begin{corollary}\icor{order of inertia group}
	We have $\# I_\p = e = e(\p/p)$.
\end{corollary}
\begin{proof}
	The exact sequence (\ref{eqn:exact}) implies that
	$\#I_\p = \#D_\p / f$ where $f = f(\p/p) = [\F_{\p} : \F_p]$.
	Applying Propositions~\ref{prop:nosplit} and \ref{prop:noresidue}, we have
	$$
		\#D_\p = \left[K : K^D\right] = \frac{[K:\Q]}{g} = \frac{efg}{g} = ef.
	$$
	Dividing both sides by $f$ proves the corollary.
\end{proof}

We have the following characterization of $I_\p$.
\begin{proposition}\label{prop:charip}\iprop{inertia group characterization}
	Let $K/\Q$ be a Galois extension with group $G$,
	and let~$\p$ be a prime of $\O_K$ lying
	over a prime~$p$.  Then
	$$
		I_\p = \left\{
		\sigma\in G \ \colon \sigma(a) \equiv a\pmod{\p}\text{ for all } a\in\O_K
		\right\}.
	$$
\end{proposition}
\begin{proof}
	By definition $I_\p = \{\sigma\in D_\p : \sigma(a) \equiv
	a\pmod{\p}\text{ for all } a\in\O_K\}$, so it suffices to show that
	if $\sigma\not\in D_\p$, then there exists $a\in\O_K$ such that
	$\sigma(a)\not\con a\pmod{\p}$.  If $\sigma\not\in D_\p$, then
	$\sigma^{-1}\not\in D_\p$, so $\sigma^{-1}(\p)\neq \p$.  Since both
	are maximal ideals, there exists $a\in\p$ with
	$a\not\in\sigma^{-1}(\p)$, i.e., $\sigma(a)\not\in\p$.  Thus
	$\sigma(a)\not\con a\pmod{\p}$.
\end{proof}

\begin{exercise}
	Let $I = I_\p$ be the inertia subgroup as above. Show that
	\begin{enumerate}
		\item $K^I$ is the largest subfield of $K$ such that $p$ is unramified
		in $K^I$.
		\item $K^I$ is the smallest subfield of $K$ such that $\p$ is totally
		ramified over $\p \cap K^I$.
	\end{enumerate}
\end{exercise}

The following diagram is mean to help summarize the relationship between $I,D$,
and the splitting of $p$ in $K$.
$$
	\begin{tikzcd}
		K \\
		K^I \\
		K^D \\
		\Q
	\end{tikzcd}
$$
[TODO]
%TODO
%\begin{figure}\label{fig}
%\includegraphics[width=\textwidth]{splitting}
%\caption{The Splitting of Behavior of a Prime in a Galois Extension}
%\end{figure}
%\begin{figure}
%	\centering
%	\begin{tikzpicture}
%		\node (Q) {$\Q$};
%		\node (KD) [above of=Q] {$K^D$};
%		\node (KI) [above of=KD] {$K^L$};
%		\node (K) [above of=KI] {$K$};
%		
%		\path[draw] (Q) edge[out=-210,in=-180] node [left] {$l_A$} (KD);
%	\end{tikzpicture}
%	\caption{Splitting of $p$ in a Galois extension $K/\Q$.}
%	\label{fig:splitting-of-p}
%\end{figure}


\section{Frobenius Elements}

Suppose that $K/\Q$ is a finite Galois extension with group $G$ and $p$ is a
prime such that $e = 1$ (i.e., an unramified prime). Then $I = I_\p = 1$ for
any $\p\mid p$, so the map $\vphi$ of Theorem~\ref{thm:redsurj} is a canonical
isomorphism $D_\p \isom \Gal(\F_{\p},\F_p)$. By Section~\ref{sec:galoisfinite},
the group $\Gal(\F_{\p},\F_p)$ is cyclic with canonical generator $\Frob_p$.
The \defn{Frobenius element} corresponding to $\p$ is $\Frob_\p \in D_\p$.
It is the unique (see Exercise~\ref{ex:frobunique}) element of $G$ such that
for all $a\in\O_K$ we have
$$
	\Frob_\p(a)\con a^p\pmod{\p}.
$$

\begin{exercise}\label{ex:frobunique}
	With the notation above, prove that $\Frob_\p$ is unique.
	That is, if $\sigma$ satisfies $\sigma(a) \equiv a^p \pmod{\p}$
	for all $a\in\O_K$ then $\sigma = \Frob_\p$.

	\begin{hint}
		First show $\sigma\in D_\p$, then argue as in
		the proof of Proposition~\ref{prop:charip}.
	\end{hint}
\end{exercise}


Just as the primes $\p$ and decomposition groups $D_\p$ are all
conjugate, the Frobenius elements corresponding to primes
$\p\mid p$ are all conjugate as elements of~$G$.

\begin{proposition}\iprop{conjugation of Frobenius}
	For each $\sigma \in G$, we have
	$$
		\Frob_{\sigma\p} = \sigma\Frob_\p\sigma^{-1}.
	$$
	In particular, the Frobenius elements lying over a given
	prime are all conjugate.
\end{proposition}
\begin{proof}
	Fix $\sigma\in G$. For any $a\in\O_K$ we have
	$\Frob_\p(\sigma^{-1}(a)) - \sigma^{-1}(a)^p \in \p$.
	Applying~$\sigma$ to both sides, we see that
	$\sigma\Frob_\p(\sigma^{-1}(a)) - a^p \in \sigma\p$,
	so $\sigma\Frob_\p\sigma^{-1} = \Frob_{\sigma \p}$.
\end{proof}

Thus the conjugacy class of $\Frob_\p$ in $G$ is a well-defined
function of~$p$.  For example, if $G$ is abelian, then $\Frob_\p$ does
not depend on the choice of $\p$ lying over $p$ and we obtain a well
defined symbol $\kr{K/\Q}{p} = \Frob_\p\in G$ called the \defn{Artin symbol}.
It extends to a homomorphism from the free abelian
group on unramified primes~$p$ to~$G$.
Class field theory (for~$\Q$) sets up a natural bijection
between abelian Galois extensions of $\Q$ and certain maps from
certain subgroups of the group of fractional ideals for~$\Z$ (i.e., $\Q^*$).
We have just described one direction of this bijection, which associates to an
abelian extension the Artin symbol (which is a homomorphism).

The Kronecker-Weber theorem asserts that the abelian extensions of
$\Q$ are exactly the subfields of the fields $\Q(\zeta_n)$, as $n$
varies over all positive integers.  By Galois theory there is a
correspondence between the subfields of the field $\Q(\zeta_n)$,
which has Galois group $(\Z/n\Z)^*$, and the subgroups of $(\Z/n\Z)^*$.
If $H \subseteq (\Z/n\Z)^*$ is the subgroup corresponding to
$K \subset \Q(\zeta_n)$ then the Artin reciprocity map
$p\mapsto \kr{K/\Q}{p}$ is given by $p\mapsto [p]\in (\Z/n\Z)^*/H$.

\begin{remark}
	Notice above that the $n$ used is not unique. That is,
	if~$K$ is an abelian extension of $\Q$ then it lies in some~$\Q(\zeta_n)$.
	But then it also lies inside of~$\Q(\zeta_{dn})$ for any
	positive integer~$d$. However, a different choice of $n$
	would mean a different choice of $H$. However the
	quotient $(\Z/n\Z)^*/H$ used is not dependent on $n$
	since it is isomorphic to the Galois group of $K/\Q$.
\end{remark}

\section{The Artin Conjecture}\label{sec:artin}

The Galois group $\Gal(\Qbar/\Q)$ is an object of central importance
in number theory, and we can interpret much of number theory as the
study of this group.  A good way to study a group is to study how it
acts on various objects, that is, to study its representations.

Endow $\galq$ with the topology which has as a basis of open neighborhoods
of the origin the subgroups $\Gal(\Qbar/K)$, where~{$K$ varies
over finite Galois extensions of~$\Q$.
Fix a positive integer~$n$ and let $\GL_n(\C)$ be the group of
$n\times n$ invertible matrices over~$\C$ with the discrete topology.

\begin{warning}\label{warn:galqtopology}
	The topology on $\galq$ is {\bf not} the topology induced
	by taking as a basis of open neighborhoods around the origin
	the collection of finite-index normal subgroups of $\galq$,
	see \cite[Ch.~7]{milne:FT} or Exercise~\ref{ex:nonopensbgpfiniteindexgalq}.
	In particular, there exist nonopen normal subgroups of finite index which
	do not correspond to subgroups $\Gal(\Qbar/K)$ for some finite Galois
	extension $K/\Q$.
\end{warning}

\begin{definition}
	A \defn{complex  $n$-dimensional representation} of $\galq$
	is a continuous homomorphism
	$$
		\rho:\galq\to \GL_n(\C).
	$$
\end{definition}
For $\rho$ to be continuous means that if $K$ is the fixed
field of $\Ker(\rho)$, then $K/\Q$ is a finite Galois extension.
We have a diagram
$$
	\begin{tikzcd}
		\galq \arrow{rr}{\rho} \arrow{dr} & & \GL_n(\C) \\
		& \Gal(K/\Q) \arrow[hook]{ur}[swap]{\rho'}
	\end{tikzcd}
$$

\begin{exercise}\label{ex:galqrepfiniteimage}
	Suppose $\rho:\galq\to\GL_n(\C)$ is continuous.
	Show that the image is finite.
\end{exercise}

\begin{remark}
	The converse to Exercise~\ref{ex:galqrepfiniteimage}
	is \textbf{false} in general (see
	Exercise~\ref{ex:nonopensbgpfiniteindexgalq}).
	This is essentially the same warning as
	Warning~\ref{warn:galqtopology}, however it is worth
	pointing out to avoid mistakes.\footnote{
	See \cite[pg.~1]{artinconjectureLectureNotes}.}
\end{remark}

\begin{exercise}\label{ex:nonopensbgpfiniteindexgalq}
	Find a nonopen subgroup of index $2$ in $\galq$.
	Note this is also an example of a non-continuous
	homomorphism $\galq\to\GL_n(\C)$ with finite image.
	
	\begin{hint}
		Use Zorn's lemma to show that there are homomorphisms
		$\galq\to\{\pm 1\}$ with finite image that are not continuous,
		since they do not factor through the Galois group of any
		finite Galois extension.
	\end{hint}
	
%	\begin{hint}
%		The extension $\Q(\sqrt{d}, d \in \Q^*/(\Q^*)^2)$
%		is an extension of~$\Q$ with Galois group $X\ncisom \prod \F_2$.
%		The index-two open subgroups of~$X$ correspond to the quadratic
%		extensions of~$\Q$. However, Zorn's lemma implies that~$X$
%		contains many index-two subgroups that do not correspond to
%		quadratic extensions of~$\Q$.
%	\end{hint}
\end{exercise}

\begin{exercise}
	Let $S_3$ by the symmetric group on three symbols, which has order $6$.
	\begin{enumerate}
		\item \label{ex:a} Observe that $S_3\isom D_3$, where $D_3$ is
		the dihedral group of order $6$, which is the group of symmetries of
		an equilateral triangle.
		\item Use (\ref{ex:a}) to write down an explicit
		embedding $S_3\hra \GL_2(\C)$.
		\item Let $K$ be the number field $\Q(\sqrt[3]{2},\omega)$,
		where $\omega^3=1$ is a nontrivial cube root of unity.  Show
		that $K$ is a Galois extension with Galois group isomorphic to~$S_3$.
		\item We thus obtain a $2$-dimensional irreducible complex
		Galois representation
		$$
		\rho:\Gal(\Qbar/\Q) \to \Gal(K/\Q)\isom S_3 \subset \GL_2(\C).
		$$
		Compute a representative matrix of $\Frob_p$ and the characteristic polynomial
		of $\Frob_p$ for $p=5,7,11,13$.
	\end{enumerate}
\end{exercise}

Fix a Galois representation~$\rho$ and let $K$ be the fixed field of
$\ker(\rho)$, so~$\rho$ factors through $\Gal(K/\Q)$.  For each prime
$p\in\Z$ that is not ramified in $K$, there is an element
$\Frob_\p\in\Gal(K/\Q)$ that is well-defined up to conjugation by
elements of $\Gal(K/\Q)$.  This means that $\rho'(\Frob_p)\in
\GL_n(\C)$ is well-defined up to conjugation.  Thus the characteristic
polynomial $F_p(x)\in\C[x]$ of $\rho'(\Frob_p)$ is a well-defined
invariant of $p$ and $\rho$.  Let
$$
	R_p(x)
	= x^{\deg(F_p)}\cdot F_p(1/x)
	= 1 + \cdots + \det(\Frob_p)\cdot x^{\deg(F_p)}
$$
be the polynomial obtain
by reversing the order of the coefficients of $F_p$.
Following E.~Artin \cite{artin:conjecture, artin:conjecture2}, set
\begin{equation}\label{eqn:artin}
	L(\rho,s) = \prod_{p\text{ unramified}}\frac{1}{R_p(p^{-s})}.
\end{equation}
We view $L(\rho,s)$ as a function of a single complex variable $s$.
One can prove that $L(\rho,s)$ is holomorphic on some right
half plane, and extends to a meromorphic function on all $\C$.
\begin{conjecture}[Artin]\label{conj:artin}
	The $L$-function of any continuous representation
	$$
		\Gal(\Qbar/\Q)\to\GL_n(\C)
	$$
	is an entire function on all $\C$, except possibly at $1$.
\end{conjecture}
This conjecture asserts that there is some way to analytically continue
$L(\rho,s)$ to the whole complex plane, except possibly at $1$.
(A standard fact from complex analysis is that this analytic
continuation must be unique.)
The simple pole at $s=1$ corresponds to the trivial representation (the
Riemann zeta function), and if $n\geq 2$ and $\rho$ is irreducible,
then the conjecture is that $\rho$ extends to a holomorphic function
on all $\C$.

The conjecture is known when $n=1$.  Assume for the rest of this
paragraph that $\rho$ is odd, i.e., if $c\in\Gal(\Qbar/\Q)$ is complex
conjugation, then $\det(\rho(c))=-1$.  When $n=2$ and the image of
$\rho$ in $\PGL_2(\C)$ is a solvable group, the conjecture is known,
and is a deep theorem of Langlands and others (see
\cite{langlands:basechange}), which played a crucial roll in Wiles's
proof of Fermat's Last Theorem.  When $n=2$ and the image of $\rho$ in
$\PGL_2(\C)$ is not solvable, the only possibility is that the
projective image is isomorphic to the alternating group~$A_5$.
Because~$A_5$ is the symmetry group of the icosahedron, these
representations are called \defn{icosahedral}.  In this case, Joe
Buhler's Harvard Ph.D. thesis \cite{buhler:thesis} gave the first
example in which $\rho$ was shown to satisfy
Conjecture~\ref{conj:artin}.  There is a book \cite{mr95i:11001},
which proves Artin's conjecture for 7 icosahedral representation (none
of which are twists of each other).  Kevin Buzzard and the author
proved the conjecture for 8 more examples \cite{buzzard-stein:artin}.
Subsequently, Richard Taylor, Kevin Buzzard, Nick Shepherd-Barron, and
Mark Dickinson proved the conjecture for an infinite class of
icosahedral Galois representations (disjoint from the examples)
\cite{bdsbt}.  The general problem for $n=2$ is in fact now completely
solved, due to recent work of Khare and Wintenberger
\cite{khare-wintenberger:serre1} that proves Serre's conjecture.

%%% Local Variables:
%%% mode: latex
%%% TeX-master: "ant"
%%% End:
