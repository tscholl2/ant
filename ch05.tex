%%%%%%%%%%%%%%%%%%%%%%%%%%%%%%%%%%%%%%%%%%%%%%%%%%%%%%%%%%%%%%%%%%%%%%%%%%
%% CRT
%%%%%%%%%%%%%%%%%%%%%%%%%%%%%%%%%%%%%%%%%%%%%%%%%%%%%%%%%%%%%%%%%%%%%%%%%%

\chapter{The Chinese Remainder Theorem}\label{ch:crt}

In this chapter, we prove the Chinese Remainder Theorem (CRT) for
arbitrary commutative rings, then apply CRT to prove that every ideal
in a Dedekind domain $R$ is generated by at most two elements.  We
also prove that $\p^n/\p^{n+1}$ is (noncanonically) isomorphic to
$R/\p$ as an $R$-module, for any nonzero prime ideal $\p$ of $R$.  The
tools we develop in this chapter will be used frequently to prove
other results later.

\section{The Chinese Remainder Theorem}

\subsection{CRT in the Integers}
The classical CRT asserts that if $n_1, \dots, n_r$ are integers that are coprime
in pairs, and $a_1, \dots, a_r$ are integers, then there exists an
integer~$a$ such that $a \equiv a_i\pmod{n_i}$ for each $i=1, \dots,r$.
Here ``coprime in pairs'' means that $\gcd(n_i,n_j)=1$ whenever
$i\neq j$; it does {\em not} mean that $\gcd(n_1, \dots, n_r)=1$,
though it implies this.
In terms of rings, CRT asserts that the
natural map
\begin{equation}\label{eqn:crt}
\ZZ/(n_1\cdots n_r)\ZZ \to (\ZZ/n_1\ZZ)\oplus \cdots \oplus (\ZZ/n_r\ZZ)
\end{equation}
that sends $a \in \ZZ$ to its reduction modulo each $n_i$,
is an isomorphism.

This map is {\em never} an isomorphism if the $n_i$ are not coprime.
Indeed, the cardinality of the image of the left hand side of
(\ref{eqn:crt}) is $\lcm(n_1, \dots, n_r)$, since it is the image of a
cyclic group and $\lcm(n_1, \dots, n_r)$ is the largest order of an
element of the right hand side, whereas the cardinality of the right
hand side is $n_1\cdots n_r$.

The isomorphism (\ref{eqn:crt}) can alternatively be viewed as
asserting that any system of linear congruences
$$
x \equiv a_1 \pmod{n_1}, \quad
x \equiv a_2 \pmod{n_2}, \quad
\dots, \quad
x \equiv a_r \pmod{n_r}
$$
with pairwise coprime moduli has a unique solution modulo $n_1\cdots n_r$.

Before proving the CRT in more generality, we prove
(\ref{eqn:crt}).
There is a natural map
$$
  \phi: \ZZ \to (\ZZ/n_1\ZZ)\oplus \cdots \oplus (\ZZ/n_r\ZZ)
$$
given by projection onto each factor.  Its kernel is
$$
 n_1 \ZZ \cap \cdots \cap n_r \ZZ.
$$
If $n$ and $m$ are integers, then $n\ZZ\cap m\ZZ$ is the
set of multiples of both $n$ and $m$, so $n\ZZ\cap m\ZZ = \lcm(n,m)\ZZ$.
Since the $n_i$ are coprime,
$$
 n_1 \ZZ \cap \cdots \cap n_r \ZZ = n_1 \cdots n_r \ZZ.
$$
Thus we have proved there is an inclusion
\begin{equation}\label{eqn:crt_inj}
 i: \ZZ/(n_1\cdots n_r)\ZZ \hookrightarrow (\ZZ/n_1\ZZ) \oplus \cdots \oplus (\ZZ/n_r\ZZ).
\end{equation}
This is half of the CRT; the other half is to prove that this map is
surjective.  In this case, it is clear that $i$ is also surjective,
because $i$ is an injective map between finite sets of the same cardinality.
We will, however, give a proof of surjectivity that doesn't use
finiteness of the above two sets.

To prove surjectivity of $i$, note that since the $n_i$ are coprime in
pairs, $$\gcd(n_1, n_2\cdots n_r)=1,$$ so there exists integers $x,y$
such that
$$
   x n_1 + y n_2\cdots n_r = 1.
$$
To complete the proof, observe that
$ y n_2\cdots n_r = 1 - x n_1$
is congruent to~$1$ modulo $n_1$ and~$0$ modulo $n_2\cdots n_r$.
Thus $(1,0, \dots,0) = i (y n_2\cdots n_r)$ is in the image of~$i$.
By a similar argument, we see that $(0,1, \dots,0)$ and the
other similar elements are all in the image of~$i$, so $i$
is surjective, which proves CRT.

\subsection{CRT in General}
Recall that {\em all rings in this book are commutative with unity}.
Let $R$ be such a ring.

\begin{definition}[Coprime]
  Ideals $I$ and $J$ of $R$ are \defn{coprime} if $I+J=(1)$.
\end{definition}

\begin{exercise}
  Let $a_1 = 1+i$, $a_2 = 3+2i$, and $a_3 = 3+4i$ as elements of $\ZZ[i]$.
  \begin{enumerate}
    \item Prove that the ideals $I_1=(a_1)$, $I_2=(a_2)$, and $I_3=(a_3)$
    are coprime in pairs.
    \item Compute the cardinality of $\ZZ[i]/(I_1 I_2 I_3)$.
    \item Find a single element in $\ZZ[i]$ that is congruent to~$n$
    modulo $I_n$, for each $n\leq 3$.
  \end{enumerate}
\end{exercise}

For example, if~$I$ and~$J$ are nonzero ideals in a Dedekind domain,
then they are coprime precisely when the prime ideals that appear in
their two (unique) factorizations are disjoint.

\begin{lemma}\label{lem:prodint}\ilem{$I\cap{}J = IJ$}
If $I$ and $J$ are coprime ideals in a ring $R$, then
$I\cap{}J = IJ$.
\end{lemma}
\begin{proof}
Choose $x\in I$ and $y\in J$
such that $x+y=1$.  If $c\in{} I\cap{} J$ then
$$c=c\cdot 1=c\cdot (x+y) = cx + cy \in IJ + IJ = IJ,$$
so $I\cap{} J\subset IJ$.
The other inclusion is obvious by the definition of an ideal.
\end{proof}

\begin{lemma}\label{lem:coprime_prod}
Suppose $I_1, \dots, I_s$ are pairwise coprime ideals.
Then $I_1$ is coprime to the product $I_2\cdots I_s$.
\end{lemma}
\begin{proof}
In the special case of a Dedekind domain, we could easily
prove this lemma using unique factorization of ideals as
products of primes (Theorem~\ref{thm:intuniqfac}); instead,
we give a direct general argument.

It suffices to prove the lemma in the case $s=3$, since the
general case then follows from induction.
By assumption, there
are $x_1 \in I_1, y_2 \in I_2$ and $a_1 \in I_1, b_3 \in I_3$
such
$$
x_1 + y_2 = 1 \qquad\text{and}\qquad a_1 + b_3 = 1.
$$
Multiplying these two relations yields
$$
x_1 a_1 + x_1 b_3 + y_2 a_1 + y_2 b_3 = 1 \cdot 1 = 1.
$$
The first three terms are in $I_1$ and the last term is in
$I_2 I_3 = I_2 \cap I_3$ (by Lemma~\ref{lem:prodint}),
so $I_1$ is coprime to $I_2 I_3$.
\end{proof}

Next we prove the general Chinese Remainder Theorem.
We will apply this result with $R=\sO_K$ in the rest of this chapter.
\begin{theorem}[Chinese Remainder Theorem]\label{thm:crt}
\ithm{chinese remainder}
Suppose $I_1, \dots, I_r$ are nonzero ideals of a ring~$R$ such
$I_m$ and $I_n$ are coprime for any $m\neq n$.  Then the natural
homomorphism $R \to \bigoplus_{n=1}^r R/I_n$ induces an isomorphism
$$
\psi: R/\prod_{n=1}^r I_n \to \bigoplus_{n=1}^r R/I_n.
$$
Thus given any $a_n \in R$, for $n=1, \dots,r$, there exists some $a\in R$
such that $a \equiv a_n\pmod{I_n}$ for $n=1, \dots, r$; moreover,~$a$
is unique modulo $\prod_{n=1}^r I_n$.
\end{theorem}

\begin{proof}
Let
$
  \varphi:R \to \bigoplus_{n=1}^r R/I_n
$
be the natural map induced by reduction modulo
the $I_n$.
An inductive application of Lemma~\ref{lem:prodint}
implies that
the kernel $\cap_{n=1}^r I_n$ of~$\varphi$
is equal to
$\prod_{n=1}^r I_n$, so the map~$\psi$ of the theorem is injective.

Each projection $R\to R/I_n$ is  surjective, so to prove
that $\psi$ is surjective, it suffices
to show that $(1,0, \dots,0)$
is in the image of~$\varphi$, and similarly for the other
factors.  By Lemma~\ref{lem:coprime_prod},
$J=\prod_{n=2}^rI_n$ is coprime to~$I_1$, so
there exists $x\in I_1$ and $y \in J$ such that
$x+y=1$.  Then $y = 1-x$ maps to~$1$ in
$R/I_1$ and to~$0$ in $R/J$, hence to~$0$ in $R/I_n$
for each $n\geq 2$, since $J\subset I_n$.
\end{proof}

\section{Structural Applications of the CRT}
Let $\sO_K$ be the ring of integers of some number field $K$, and
suppose~$I$ is a nonzero ideal of $\sO_K$.  As an abelian group $\sO_K$
is free of rank $[K:\QQ]$, and~$I$ is of finite index in $\sO_K$, so~$I$
is generated by $[K:\QQ]$ generators as an abelian group, so as an
$R$-ideal $I$ requires at most $[K:\QQ]$ generators.  The main result
of this section asserts something better, namely that~$I$ can be
generated {\em as an ideal} by at most two elements.  Moreover, our
result is more general, since it applies to an arbitrary Dedekind
domain $R$. Thus, for the rest of this section, $R$ is any Dedekind
domain, e.g., the ring of integers of either a number field or
function field.  We use CRT to prove that every ideal of $R$ can be
generated by two elements.

\begin{warning}
If we replace $R$ by an order in a Dedekind domain, i.e.,
by a subring of finite index, then there may be ideals that
require far more than $2$ generators.
\end{warning}

Suppose that~$I$ is a nonzero integral ideal of
$R$.  If $a\in I$, then $(a)\subset I$, so~$I$ divides~$(a)$ and
the quotient $(a)I^{-1}$ is an integral ideal.  The following lemma
asserts that~$(a)$ can be chosen so the quotient $(a)I^{-1}$ is coprime to
any given ideal.
\begin{lemma}\label{lem:magica}
If $I$ and $J$ are nonzero integral ideals in $R$, then there exists
an $a\in I$ such that the integral ideal $(a)I^{-1}$ is coprime to~$J$.
\end{lemma}
Before we give the proof in general, note that the lemma is trivial
when $I$ is principal, since if $I=(b)$, just take $a=b$, and
then $(a)I^{-1} = (a)(a^{-1})= (1)$ is coprime to every ideal.
\begin{proof}
Let $\p_1, \dots, \p_r$ be the prime divisors of~$J$.
For each $n$, let $v_n$ be the largest power of $\p_n$
that divides~$I$.
Since $\p_n^{v_n}\neq \p_n^{v_n+1}$,
we can choose an element $a_n\in \p_n^{v_n}$
that is not in $\p_n^{v_n+1}$.
By Theorem~\ref{thm:crt} applied to
the $r+1$ coprime integral ideals
$$
  \p_1^{v_1+1}, \dots, \p_r^{v_r+1}, \, I\cdot \left(\prod \p_n^{v_n}\right)^{-1},
$$
there exists $a\in R$
such that
$$
   a \equiv a_n \pmod{\p_n^{v_n+1}}
$$
for all $n=1, \dots, r$ and
also
$$
   a \equiv 0 \quad \left(\text{mod}\,\,\, I\cdot \left(\prod \p_n^{v_n}\right)^{-1}\right).
$$

To complete the proof we show that $(a)I^{-1}$ is not
divisible by any $\p_n$, or equivalently, that each
$\p_n^{v_n}$ exactly divides $(a)$.
First we show that $\p_n^{v_n}$ divides $(a)$. Because
$a\equiv a_n \pmod{\p_n^{v_n+1}}$, there exists
$b \in \p_n^{v_n+1}$ such that $a = a_n + b$.  Since
$a_n\in \p_n^{v_n}$ and $b \in \p_n^{v_n + 1} \subset \p_n^{v_n}$,
it follows that $a\in \p_n^{v_n}$,
so $\p_n^{v_n}$ divides~$(a)$.
Now assume for the sake of contradiction that
$\p_n^{v_n+1}$ divides $(a$); then $a_n=a-b\in \p_n^{v_n+1}$, which
contradicts that we chose $a_n \not\in\p_n^{v_n+1}$.
Thus
$\p_n^{v_n+1}$ does not divide $(a)$, as claimed.
\end{proof}



\begin{proposition}\iprop{ideals generated by two elements}\label{prop:2gen}
Suppose $I$ is a fractional ideal in a Dedekind domain $R$.  Then there exist $a,b\in{}K$ such that
$I=(a,b)=\{\alpha a + \beta b : \alpha,\beta \in R\}$.
\end{proposition}
\begin{proof}
If $I=(0)$, then $I$ is generated by $1$ element and we are done.  If
$I$ is not an integral ideal, then there is an $x\in K$ such that $xI$ is
an integral ideal, and the number of generators of $xI$ is the same as
the number of generators of $I$, so we may assume that $I$ is an
integral ideal.

Let $a$ be {\em any} nonzero element of the integral ideal~$I$.  We
will show that there is some $b\in I$ such that $I=(a,b)$.  Let
$J=(a)$.  By Lemma~\ref{lem:magica}, there exists $b\in I$ such that
$(b)I^{-1}$ is coprime to $(a)$.  Since $a,b\in I$, we have $I\mid
(a)$ and $I\mid (b)$, so $I\mid (a,b)$.  Suppose $\p^n\mid (a,b)$ with
$\p$ prime and $n\geq 1$.  Then $\p^n\mid (a)$ and $\p^n\mid (b)$, so
$\p\nmid (b)I^{-1}$, since $(b)I^{-1}$ is coprime to $(a)$.  We have
$\p^n\mid(b) = I\cdot (b)I^{-1}$ and $\p\nmid (b)I^{-1}$, so $\p^n
\mid I$.  Thus by unique factorization of ideals in $R$ we
have that $(a,b)\mid I$.  Since $I \mid (a,b)$ we conclude
that  $I=(a,b)$, as claimed.
\end{proof}

We can also use Theorem~\ref{thm:crt} to determine the
$R$-module structure of $\p^n/\p^{n+1}$.
\begin{proposition}\label{prop:quopow}\iprop{structure of $\p^n/\p^{n+1}$}
Let $\p$ be a nonzero prime ideal of $R$, and let $n\geq 0$ be an
integer.  Then $\p^n/\p^{n+1} \isom R/\p$ as $R$-modules.
\end{proposition}
\begin{proof}[Proof~\footnote{Proof from \cite[pg.~13]{sd:brief}.}]
Since $\p^n\neq \p^{n+1}$, by unique factorization,
there is an element $b\in
\p^n$ such that $b\not\in \p^{n+1}$.  Let
$\varphi:R\to\p^n/\p^{n+1}$ be the $R$-module morphism defined by
$\varphi(a)=ab$.  The kernel of $\varphi$ is $\p$ since clearly
$\varphi(\p)=0$ and if $\varphi(a)=0$ then $ab\in\p^{n+1}$, so
$\p^{n+1}\mid (a)(b)$, so $\p \mid (a)$, since $\p^{n+1}$ does not
divide~$(b)$.  Thus~$\varphi$ induces an injective $R$-module
homomorphism $R/\p \hookrightarrow \p^{n}/\p^{n+1}$.

It remains to show that $\varphi$ is surjective, and this is where we
will use Theorem~\ref{thm:crt}.   Suppose $c\in \p^{n}$.
By Theorem~\ref{thm:crt} there exists $d\in R$
such that
$$
  d \equiv c\pmod{\p^{n+1}}
\qquad\text{and}\qquad
  d \equiv 0\pmod{(b)/\p^{n}}.
$$
We have $\p^n\mid (d)$ since $d\in\p^n$ and $(b)/\p^n\mid (d)$
by the second displayed condition, so
since $\p\nmid(b)/\p^n$, we have $(b)=\p^n\cdot(b)/\p^n\mid (d)$, hence
$d/b\in R$.   Finally
\[
 \varphi\left(\frac{d}{b}\right) \quad \equiv \quad \frac{d}{b}\cdot b \pmod{\p^{n+1}}
 \quad \equiv \quad d\pmod{\p^{n+1}} \quad\equiv \quad c\pmod{\p^{n+1}},
\]
so $\varphi$ is surjective.
\end{proof}

\begin{exercise}\label{ex:residuefieldofpower}(See \cite[Thm.~22(a)]{marcus1977number}) %pg 67
  Let $R$ be a Dedekind domain and $\p$ a nonzero prime ideal in $R$.
  Show that $\#(R/\p^m) = \#(R/\p)^m$.

  Note: $\#(R/\p)$ is not finite in general! For example,
  The ring of formal power series $k[[t]]$ for some field $k$
  is a Dedekind domain and the residue field at the prime $(t)$
  is $k$.

  \begin{hint}
    Consider the exact sequence
    $$
      0\to \p/\p^{m} \to R/\p^{m} \to R/\p^{m-1} \to 0
    $$
    and the chain
    $$
      \p^m \subseteq \p^{m-1}
      \subseteq \cdots \subseteq \p^2 \subseteq \p.
    $$
  \end{hint}
\end{exercise}

\begin{remark}
  There is one special case of the previous exercise that you probably
  have seen before: the size of $\ZZ/4\ZZ$ is the same as
  $(\ZZ/2\ZZ)^2$. In fact you might have seen a proof of
  the fact that $\ZZ/n^m\ZZ$ has the same cardinality as $\left(\ZZ/n\ZZ\right)^m$
  in a standard group theory or abstract algebra course.
\end{remark}

\section{Computing Using the CRT}
In order to explicitly compute an $a$ as given by Theorem~\ref{thm:crt},
usually one first precomputes elements $v_1, \dots, v_r \in R$ such that
$v_1 \mapsto (1,0, \dots, 0)$,
$v_2 \mapsto (0,1, \dots, 0)$, etc.
Then given any $a_n \in R$, for $n=1, \dots, r$, we obtain an $a \in R$
with $a_n \equiv a\pmod{I_n}$ by taking
$$
  a = a_1 v_1 + \cdots + a_r v_r.
$$
How to compute the $v_i$ depends on the ring~$R$.   It reduces to
the following problem: Given coprimes ideals $I,J \subset R$, find
$x\in I$ and $y\in J$ such that $x+y=1$.   If $R$ is torsion free and
of finite rank
as a $\ZZ$-module, so $R\ncisom \ZZ^n$,
then $I, J$ can be represented by giving a basis in terms of a basis
for~$R$, and finding $x,y$ such  that $x+y=1$ can then be reduced to
a problem in linear algebra over~$\ZZ$.
More precisely, let~$A$
be the matrix whose columns are the concatenation of a basis for~$I$
with a basis for~$J$.
Suppose $v\in \ZZ^n$ corresponds to $1\in\ZZ^n$.
Then finding $x,y$ such that $x+y=1$ is equivalent to
finding a solution $z\in \ZZ^n$ to the matrix equation
$Az = v$. This latter linear algebra problem
can be solved using  \index{Smith normal form} or \index{Hermite normal form}
(see \cite[\S4.7.1]{cohen:course_ant}),
which is a generalization over $\ZZ$
of reduced row echelon form.

Next we give an explicit example of a CRT computation using {\Sage}. Let $K = \QQ(\sqrt{-1})$ and $R = \sO_K$. We will set $I = (1 + i)$ and $J = (3)$.
\begin{sagecode}
\begin{sagecell}
K.<i> = QuadraticField(-1)
d = K.degree()
I = K.ideal(1 + i)
J = K.ideal(3)
\end{sagecell}
\end{sagecode} %link

Number fields in {\Sage} come with a $\QQ$-vector space isomorphism $K \to \QQ^d$,
where $d = \deg K$.
To turn an element $\alpha \in K$ into a vector, we use the {\tt vector()} method.
We can build the matrix $A$ described above as follows.
\begin{sagecode} %link
\begin{sagecell}
rows = [x.vector() for x in I.basis() + J.basis()]
A = Matrix(ZZ,rows).transpose()
\end{sagecell}
\end{sagecode} %link
Next we compute the Smith normal form $S$ of $A$,
along with matrices $T,U$ such that $S = TAU$.
\begin{sagecode} %link
\begin{sagecell}
S,T,U = A.smith_form(transformation=True)
\end{sagecell}
\end{sagecode} %link

We have the following chain of $\ZZ$-linear maps
$$
	\ZZ^{2d} \xrightarrow{U} \ZZ^{2d} \xrightarrow{A} \ZZ^d \xrightarrow{T} \ZZ^d.
$$
The matrix $S$ represents the composition.
The cokernel of matrix $A$ is trivial since $\sO_K/(I + J) = 0$.
Therefore $S$ is of the form $\begin{pmatrix} I_d & 0 \end{pmatrix}$
(see Section~\ref{sec:fg}).
In particular, $SS^t = I_d$\todo{check transpose syntax}.
So we can find a solution to $Az = v$ for any $v \in \ZZ^d$
by computing $z = US^tTv$. Then $Az = AUS^tTv = T^{-1}SU^{-1}US^tTv = v$.

Next we find the solution $z$ for the equation $Az = v$
where the vector $v$ is the vector corresponding to $1$.
\begin{sagecode} %link
\begin{sagecell}
v = K(1).vector()
z = T*S.transpose()*U*K(1)
\end{sagecell}
\end{sagecode} %link
Recall that the first half of the columns of $A$ represent a basis for $I$,
and the second half represents a basis for $J$.
Using the entries of {\tt z} as coefficients, we can find elements
$x \in I$ and $y \in J$ such that $x + y = 1$.
\begin{sagecode} %link
\begin{sagecell}
x = sum(z[i]*I.basis()[i] for i in range(d))
y = sum(z[d+i]*J.basis()[i] for i in range(d))
print x + y
\end{sagecell}
\begin{sageout}
1
\end{sageout}
\end{sagecode} %link

Our value of $x$ and $y$ can be used to solve for $a \in \sO_K$ such that
$a \equiv a_1 \pmod{I}$ and $a \equiv a_2 \pmod{J}$ for any given $a_1,a_2$.
We demonstrate this with $a_1 = 17 + i$ and $a_2 = 2 + 11i$.
\begin{sagecode} %link
\begin{sagecell}
a1 = 17 + i
a2 = 2 + 11*i
a = x*a2 + y*a1
print (a - a1 in I) and (a - a2 in J)
\end{sagecell}
\begin{sageout}
True
\end{sageout}
\end{sagecode}



\begin{comment}

\subsection{{\Sage}}

%We next describe how to use \magma{} and PARI to do CRT computations.

[[TODO]]

\subsection{\magma{}}
The \magma{} command {\tt ChineseRemainderTheorem} implements the
algorithm suggested by Theorem~\ref{thm:crt}.  In the following example,
we compute a prime over~$(3)$ and a prime over~$(5)$ of the ring of
integers of $\QQ(\sqrt[3]{2})$, and find an element of $\sO_K$ that is
congruent to $\sqrt[3]{2}$ modulo one prime and~$1$ modulo the other.
\begin{verbatim}
   > R<x> := PolynomialRing(RationalField());
   > K<a> := NumberField(x^3-2);
   > OK := RingOfIntegers(K);
   > I := Factorization(3*OK)[1][1];
   > J := Factorization(5*OK)[1][1];
   > I;
   Prime Ideal of OK
   Two element generators:
       [3, 0, 0]
       [4, 1, 0]
   > J;
   Prime Ideal of OK
   Two element generators:
       [5, 0, 0]
       [7, 1, 0]
   > b := ChineseRemainderTheorem(I, J, OK!a, OK!1);
   > K!b;
   -4
   > b - a in I;
   true
   > b - 1 in J;
   true
\end{verbatim}

\subsection{PARI}
There is also a CRT algorithm for number fields in PARI, but it
is more cumbersome to use.  First we defined $\QQ(\sqrt[3]{2})$
and factor the ideals $(3)$ and $(5)$.
\begin{verbatim}
   ? f = x^3 - 2;
   ? k = nfinit(f);
   ? i = idealfactor(k,3);
   ? j = idealfactor(k,5);
\end{verbatim}

Next we form matrix whose rows correspond to a product of two primes,
one dividing $3$ and one dividing $5$:
\begin{verbatim}
   ? m = matrix(2,2);
   ? m[1,] = i[1,];
   ? m[1,2] = 1;
   ? m[2,] = j[1,];
\end{verbatim}
Note that we set {\tt m[1,2] = 1}, so the exponent is 1
instead of $3$.
We apply the CRT to obtain a lift in terms
of the basis for $\sO_K$.
\begin{verbatim}
   ? ?idealchinese
   idealchinese(nf,x,y): x being a prime ideal factorization and y
   a vector of elements, gives an element b such that
   v_p(b-y_p)>=v_p(x) for all prime ideals p dividing x,
   and v_p(b)>=0 for all other p.
   ? idealchinese(k, m, [x,1])
   [0, 0, -1]~
   ? nfbasis(f)
   [1, x, x^2]
\end{verbatim}
Thus PARI finds the lift $-(\sqrt[3]{2})^2$, and we finish by
verifying that this lift is correct. The
{\tt idealval} function returns the number of times
a prime appears in the factorization of an ideal. We will use it
to check that $-(\sqrt[3]{2})^2 - \sqrt[3]{2}$ is contained in
the prime above $3$ and $-(\sqrt[3]{2})^2 - 1$ is contained in
the prime above $5$.
\begin{verbatim}
   ? idealval(k,-x^2 - x,i[1,1])
   1
   ? idealval(k,-x^2 - 1,j[1,1])
   1
\end{verbatim}
\end{comment}