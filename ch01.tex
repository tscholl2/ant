%%%%%%%%%%%%%%%%%%%%%%%%%%%%%%%%%%%%%%%%%%%%%%%%%%%%%%%%%%%%%%%%%%%%%%%%%%
%% INTRO
%%%%%%%%%%%%%%%%%%%%%%%%%%%%%%%%%%%%%%%%%%%%%%%%%%%%%%%%%%%%%%%%%%%%%%%%%%

\chapter{Introduction}

\section{Mathematical background}

In addition to general mathematical maturity,
this book assumes you have the following background:
\begin{itemize}
  \item Basics of finite group theory
  \item Commutative rings, ideals, quotient rings
  \item Some elementary number theory
  \item Basic Galois theory of fields
  \item Point set topology
  \item Basics of topological rings, groups, and measure theory
\end{itemize}
For example, if you have never worked with finite groups before, you
should read another book first. If you haven't seen much elementary
ring theory, there is still hope, but you will have to do some
additional reading and exercises.  We will briefly review the basics of
the Galois theory of number fields.

Some of the homework problems involve using a computer, but there
are examples which you can build on.  We will not assume that you have
a programming background or know much about algorithms. Most
of the book uses {\Sage} (\url{http://sagemath.org}), which is
free open source mathematical software.  The following is an example
{\Sage} session:
\begin{sagecode}
\begin{sagecell}
2 + 2
\end{sagecell}
\begin{sageout}
4
\end{sageout}
\begin{sagecell}
k.<a> = NumberField(x^2 + 1); k
\end{sagecell}
\begin{sageout}
Number Field in a with defining polynomial x^2 + 1
\end{sageout}
\end{sagecode}

\section{What is algebraic number theory?}

A number field $K$ is a finite degree extension of the rational
numbers $\QQ$.  The primitive element theorem from field theory
asserts that every such extension can be represented as the set of all
polynomials of degree less than $d = [K:\QQ] = \dim_{\QQ} K$ in
a single root $\alpha$ of some polynomial with coefficients in $\QQ$:
$$
  K = \QQ(\alpha) = \left\{ \sum_{n=0}^{d-1} a_n \alpha^n : a_n\in\QQ \right\}.
$$

Note that
$\QQ(\alpha)$ is non-canonically isomorphic to $\QQ[x]/(f)$, where $f$
is the minimal polynomial of~$\alpha$ over $\QQ$.
The homomorphism $\QQ[x]\to\QQ(\alpha)$ that sends~$x$ to~$\alpha$
has kernel~$(f)$, hence it induces an isomorphism between
$\QQ[x]/(f)$ and $\QQ(\alpha)$.
It is not canonical, since $\QQ(\alpha)$ could
have nontrivial automorphisms.  For example, if $\alpha=\sqrt{2}$, then
$\QQ(\sqrt{2})$ is isomorphic as a field to $\QQ(-\sqrt{2})$ via
$\sqrt{2}\mapsto -\sqrt{2}$.  There are two isomorphisms
$\QQ[x]/(x^2-2)\to \QQ(\sqrt{2})$.


\defn{Algebraic number theory} is the study of number fields, their rings of
integers, and related objects (e.g., functions fields, elliptic curves, etc.).
To gain a deeper understanding of these concepts, one uses techniques from
(mostly commutative) algebra and finite group theory.
The main objects that we study in this book are number fields, rings of
integers of number fields, unit groups, ideal class groups, norms, traces,
discriminants, prime ideals, Hilbert and other class fields and
associated reciprocity laws, zeta and $L$-functions, and algorithms
for computing with each of the above.

\subsection{Topics in this book}
These are some of the main topics that are discussed in this book:
\begin{itemize}\setlength{\itemsep}{-.7ex}
  \item Rings of integers of number fields
  \item Unique factorization of nonzero ideals in Dedekind domains
  \item Structure of the group of units of the ring of integers
  \item Fractional ideals and class groups
  \item Decomposition and inertia groups, Frobenius elements
  \item Ramification
  \item Discriminant and different
  \item Quadratic and biquadratic fields
  \item Cyclotomic fields (and applications)
  \item How to use {\Sage} to compute with many of the above objects
\end{itemize}
We will also touch on elliptic curves and $L$-functions.
However we will not do anything nontrivial with these subjects.


\section{Some applications of algebraic number theory}
The following examples illustrate some of the power, depth, and
importance of algebraic number theory.

\begin{description}
\item[Integer factorization:]
The number field sieve is the asymptotically fastest known algorithm for
factoring general large integers (that don't have too special of a
form).  On December 12, 2009, the number field sieve was used to
factor the RSA-768 challenge, which is a 232 digit number that is a
product of two primes:
\begin{sagecode}
\begin{sagecell}
rsa768 = 12301866845301177551304949583849627207728535695\
95334792197322452151726400507263657518745202199786469389\
95647494277406384592519255732630345373154826850791702612\
21429134616704292143116022212404792747377940806653514195\
97459856902143413
n = 3347807169895689878604416984821269081770479498371376\
85689124313889828837938780022876147116525317430877378144\
67999489
m = 3674604366679959042824463379962795263227915816434308\
76426760322838157396665112792333734171433968102700927987\
36308917
n*m == rsa768
\end{sagecell}
\begin{sageout}
True
\end{sageout}
\end{sagecode}
This record integer factorization cracked a certain 768-bit public key
cryptosystem (see \cite{cryptoeprint:2010:006}), thus
establishing a lower bound on one's choice of key size:
\begin{sagecode}
\begin{sagecell}[language=bash]
\$ man ssh-keygen   # in ubuntu-12.04
...
     -b bits
             Specifies the number of bits in the key to
             create. For RSA keys, the minimum size is
             768 bits ...
\end{sagecell}
\end{sagecode}

\item[Primality testing:]
Agrawal and his students Saxena and
Kayal found in 2002 the first ever deterministic
polynomial-time (in the number of digits) primality test
\cite{agrawal2004primes}. Their methods involve arithmetic
in quotients of $(\ZZ/n\ZZ)[x]$, which are best understood in
the context of algebraic number theory.

\item[Deeper point of view:]
Some questions in number theory
are best viewed from the point of view of algebraic number
theory such as:
\begin{itemize}
  \item
    Pell's Equation $x^2-dy^2=1$ can be reinterpreted
    in terms of units in real quadratic fields, which
    leads to a study of unit groups of number fields.
  \item
    Integer factorization is a special case of factoring nonzero
    ideals in rings of integers of number fields.
  \item
    The Riemann hypothesis about the zeros of $\zeta(s)$
    generalizes to zeta functions of number fields.
  \item
    Reinterpreting Gauss's quadratic reciprocity law in terms of
    the arithmetic of cyclotomic fields $\QQ(e^{2\pi i/n})$ leads
    to class field theory, which in turn leads to the Langlands
    program.
\end{itemize}

\item[Fermat's Last Theorem:]
This classical theorem says $x^n+y^n=z^n$ has no solutions with
$x,y,z,n$ all positive integers and $n \geq 3$. Wiles's
proof of Fermat's Last Theorem uses methods from
algebraic number theory extensively, in addition to many other deep
techniques.  Attempts to prove Fermat's Last Theorem long ago were
hugely influential in the development of algebraic number theory
by Dedekind, Hilbert, Kummer, Kronecker, and others.

\item[Arithmetic geometry:] This is a huge field that studies
solutions to polynomial equations that lie in arithmetically
interesting rings, such as the integers or number fields.  A
major triumph of arithmetic geometry is Faltings's proof of Mordell's
Conjecture.
\begin{theorem}[Faltings]\label{thm:faltings}\ithm{Faltings}
  Let $X$ be a nonsingular plane algebraic curve over a number
  field $K$.  Assume that the manifold $X(\CC)$ of complex solutions to
  $X$ has genus at least $2$ (i.e., $X(\CC)$ is topologically a donut
  with at least two holes).  Then the set $X(K)$ of points on $X$ with
  coordinates in~$K$ is finite.
\end{theorem}
For example, Theorem~\ref{thm:faltings} implies that for any $n\geq 4$
and any number field~$K$, there are only finitely many solutions
in~$K$ to $x^n+y^n=1$.

A major open problem in arithmetic geometry is the
\defn{Birch and Swinnerton-Dyer conjecture}.
An \defn{elliptic curve} $E$ is an algebraic curve with at
least one point
with coordinates in $K$ such that the set of complex points
$E(\CC)$ is a topological torus (i.e., $E(\CC)$ is topologically a donut
with one hole).
The Birch and Swinnerton-Dyer conjecture gives a
criterion for whether or not $E(K)$ is infinite in
terms of analytic properties of the $L$-function $L(E,s)$. See
\url{http://www.claymath.org/millennium/Birch_and_Swinnerton-Dyer_Conjecture/}.

\end{description}
