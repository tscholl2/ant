\chapter{Galois Cohomology}\label{ch:gc}

Let $G$ be a group and suppose $G$ acts on an abelian group $A$
(defined below). In this chapter we will study abelian groups attached
to the action of $G$ on $A$. These are called \emph{cohomology~groups}
and denoted by $\H^n(G,A)$. The theory of these groups is referred
to as \emph{group cohomology}. In the later sections $G$ will represent
the Galois group of a field extension. This is called
\emph{Galois~cohomology}. Studying Galois cohomology helps us
understand the structure of Galois groups such as $\Gal(\Qbar/\Q)$.

\section{Group Rings and Modules}

In this section we define group modules, which are analogous
to modules over a ring. For a review of the theory of modules
over a ring see \cite[Ch.~10]{dummit2004abstract}.

\begin{definition}\label{def:groupring}
	Let $G$ be any group. The \emph{group ring} $\Z[G]$ of $G$
	is the free abelian group (equivalently the free $\Z$-module) on the elements of $G$ equipped
	with multiplication given by the group structure on~$G$.
	Note that $\Z[G]$ is a commutative ring if and only if~$G$ is
	abelian.
\end{definition}

\begin{example}
	For example, the group ring of the cyclic group
	$C_n=\langle a\rangle$ of order~$n$ is
	the free $\Z$-module on $1,a,\ldots, a^{n-1}$, and the multiplication
	is induced by $a^i a^j = a^{i+j} = a^{i + j \pmod{n}}$ extended
	linearly. For example, in  $\Z[C_3]$ we have
	$$
	(1 + 2 a)(1 - a^2) = 1 - a^2 + 2a - 2 a^3
	= 1 + 2a - a^2 - 2 = -1 + 2a - a^2.
	$$
	Since $a^3 = 1$
	you might think that $\Z[C_3]$ is isomorphic to the ring $\Z[\zeta_3]$
	of integers of $\Q(\zeta_3)$, but you would be wrong, since the ring
	of integers is isomorphic to $\Z^2$ as an abelian group, but $\Z[C_3]$
	is isomorphic to $\Z^3$ as abelian group. Note that $\Q(\zeta_3)$
	is a quadratic extension of~$\Q$.
\end{example}

\begin{exercise}
	Is $\Z[\zeta_3]$ isomorphic to the group ring of some group?

	Hint: Note that the rank of the group ring as a
	$\Z$-module is equal to the size of the group.
	If $\Z[\zeta_3]$ was a group ring then it would
	have to be isomorphic to $\Z[C_2]$.
\end{exercise}
%Solution: no. Suppose you had an isomorphism and derive a
%contradiction

\begin{exercise}
	\hfill
	\begin{enumerate}
		\item[(a)]
			Write down any two elements of $\Z[\Z]$ and multiply them.
			This is not hard, but is good practice with the concept
			of a group ring.
		\item[(b)]
			Show $\Z[\Z]$ is isomorphic to $\Z\left[x,\frac{1}{x}\right]$.
	\end{enumerate}
\end{exercise}

\begin{definition}
	Let $G$ be a finite group. A \emph{$G$-module} is
	an abelian group $A$ equipped with a left action of~$G$,
	i.e., a group homomorphism $G\to\Aut(A)$, where $\Aut(A)$
	denotes the group of group isomorphisms $A\to A$ with
	the operation of function composition.
\end{definition}

\begin{exercise}\label{ex:equivalentdata}
	Fix an abelian group $A$.
	Show the following are equivalent sets of data.
	Specifically, given any one of the following objects,
	there is a natural way to construct another.
	\begin{enumerate}[label=(\alph*)]
		\item\label{itm:actionashom} A group homomorphism $G\to \Aut(A)$.
		\item\label{itm:actionasmap}
			A map $\rho:G\times A \to A$ such that
			for all $g,h\in G$ and $a,b\in A$,
			\begin{enumerate}[label=(\roman*)]
				\item
					$\rho(g,a+b) = \rho(g,a) + \rho(g,b)$
				\item
					$\rho(e,a) = a$ where $e$ is the identity in $G$.
				\item
					$\rho(gh,a) = \rho(g,\rho(h,a))$
			\end{enumerate}
		\item\label{itm:actionasringhom}
			A ring homomorphism $\Z[G] \to \End(A)$.
		\item\label{itm:actionasringmap}
			A map $\rho:\Z[G]\times A \to A$ with
			the same properties listed in \ref{itm:actionasmap}.
	\end{enumerate}
\end{exercise}

\begin{remark}
	In Exercise~\ref{ex:equivalentdata}, part \ref{itm:actionashom}
	is our definition
	of a $G$-module and parts \ref{itm:actionasringhom}
	and \ref{itm:actionasringmap} are
	the data of a $\Z[G]$-module. This shows that a $G$-module
	in the above sense is the same as a $\Z[G]$-module
	in the usual module sense.
\end{remark}

\begin{example}
	If $G$ is any finite group and $A$ any abelian group
	then we can always make $A$ into a $G$-module by
	giving it the trivial action.
	In particular, $\Z$ with the trivial action is a
	module over any group~$G$, as is $\Z/m\Z$ for any positive
	integer~$m$. Another example is $G=(\Z/n\Z)^*$, which acts
	via multiplication on $A = \Z/n\Z$.
\end{example}

\begin{remark}
	The construction $\Z[G]$ from $G$ is natural, in the
	sense that it defines a functor between categories.
	Moreover, $\Z[G]$ is the most natural way to construct
	a ring from a group in the sense that the group ring
	functor is a left adjoint to the forgetful functor from
	rings to groups. These types of functors are sometimes
	called ``free'' functors. If you are interested in
	free objects, see if you can come up with a natural way
	to add structure to other objects. Could you make a set
	into a group? How about a vector space?
\end{remark}

\section{Group Cohomology}

Let $G$ be a finite group and $A$ a $G$-module.
For each integer $n\geq 0$ there is an abelian group $\H^n(G,A)$
called the \emph{$n$th cohomology group of~$G$ acting on~$A$}.  The
general definition is somewhat complicated, but the definition for
$n\leq 1$ is fairly concrete.
For example, the \emph{$0$th cohomology group}
$$
	\H^0(G,A) = \{x \in A : \sigma x = x \text{ for all } \sigma \in G\} = G^A
$$
is the subgroup of elements of $A$ that are fixed by every element
of~$G$.

The \emph{first cohomology group}
$$
	\H^1(G,A) = C^1(G,A)/B^1(G,A)
$$
is the group $C^1$ of \emph{$1$-cocycles} modulo the group $B^1$ of
\emph{$1$-coboundaries}, where
$$
	C^1(G, A) = \{f : G \to A \text{ such that } f(\sigma\tau)
	= f(\sigma) + \sigma f(\tau)\}
$$
where the maps $f:G\to A$ range over all set-theoretic maps.
If we let $f_a: G \to A$ denote the set-theoretic map $f_a(\sigma) = \sigma(a)-a$,
then
$$
	B^1(G, A) = \{f_a :  a\in A\}.
$$
There are also explicit, and increasingly complicated, definitions of
$\H^n(G,A)$ for each $n\geq 2$ in terms of \emph{crossed homomorphisms},
which are certain maps $G \times \cdots \times G \to A$ modulo a subgroup.
We will not need these maps, but for more information about them
see \cite[Ch.~IV.2]{cassels-frohlich}.

\begin{exercise}\label{ex:H1hom}
	Suppose $G$ acts trivially on $A$.
	Show that $B^1(G,A)=0$ and $C^1(G,A) \cong \Hom(G,A)$.
	In particular, this shows $\H^1(G,A) \cong \Hom(G,A)$.
	Deduce that if $A = \Z$ then $\H^1(G,\Z) = 0$.
	Here $\Hom(G,A)$ represents the set of group homomorphisms
	from $G$ to $A$. It comes with a natural group structure
	given by $(f_1+f_2)(a) = f_1(a)+f_2(a)$.

	\begin{hint}
		For any $\sigma\in G$ we have
		$f_a(\sigma) = \sigma(a) - a = a - a = 0$.
		Also for any finite group $G$, show that $\Hom(G,\Z) = 0$.
	\end{hint}
% solution:
% $\sigma a - a = a -a =0$ for any $a\in A$.
% Also, $C^1(G,A) = \Hom(G,A)$.
% If $A=\Z$, then since $G$ is finite there are no nonzero
% homomorphisms $G\to \Z$, so $\H^1(G,\Z)=0$.
\end{exercise}

\begin{example}
	The groups $H^n(G,\Z)$ and $H^n(G,\Z/p\Z)$ (where $p$ is a prime)
	are computable in \sage. For example we can compute $H^{10}(A_5,\Z)$
	and $H^{7}(A_5,\Z/5\Z)$ where $A_5$ is the alternating group of
	order $120$ and $\Z/5\Z$ is given the trivial $A_5$-module structure.
\begin{sagecode}
\begin{sagecell}
G = AlternatingGroup(5); G
\end{sagecell}
\begin{sageout}
Alternating group of order 5!/2 as a permutation group
\end{sageout}
\begin{sagecell}
G.cohomology(10)
\end{sagecell}
\begin{sageout}
Multiplicative Abelian group isomorphic to C2 x C2
\end{sageout}
\begin{sagecell}
G.cohomology(7,5)
\end{sagecell}
\begin{sageout}
Multiplicative Abelian group isomorphic to C5
\end{sageout}
\end{sagecode}
\end{example}

\subsection{The Main Theorem}

\begin{definition}
	If $X$ is any abelian group, then $A = \Hom(\Z[G], X)$
	is a $G$-module, see Exercise~\ref{excer:group-action-on-homs}.
	We call a module constructed in this way \emph{coinduced}.
\end{definition}

\begin{exercise}\label{excer:group-action-on-homs}
	Let $X$ be any abelian group. Show that $A = \Hom(\Z[G],X)$
	is a $G$-module with the action induced by $(g\cdot f)(h) = f(hg)$
	for all $g\in G$, $f\in \Hom(\Z[G],X)$, and $h\in \Z[G]$.
\end{exercise}
%\begin{solution}
%	Note
%	$$
%		(g \cdot f)\left(\sum n_hh\right) = \sum n_hf(hg)
%	$$
%	It's an additive homomorphism
%	$$
%		g\cdot (f_1+f_2)(h) = (f_1+f_2)(hg) = f_1(hg) + f_2(hg)
%		= (g\cdot f_1 + g\cdot f_2)(h).
%	$$
%	If $g$ is the identity then clearly
%	$$
%		g\cdot f = f.
%	$$
%	And it's associative
%	$$
%		(g_1g_2 \cdot f)(h) = f(hg_1g_2)
%		= (g_2\cdot f)(hg_1) = (g_1\cdot (g_2\cdot f))(h)
%	$$
%\end{solution}


The following theorem gives three properties of group cohomology,
which uniquely determine group cohomology.
\begin{theorem}\label{thm:cohomology}
	Suppose $G$ is a finite group.  Then
	\begin{enumerate}
		\item
			We have $\H^0(G,A) = A^G$.
		\item
			If $A$ is a coinduced $G$-module,
			then $\H^n(G,A) = 0$ for all $n\geq 1$.
		\item
			If $0\to A \to B \to C \to 0$ is any exact sequence of
			$G$-modules, then there is a long exact sequence
	\end{enumerate}
	$$
	\begin{tikzcd}
0 \rar & \H^0(G,A) \rar & \H^0(G,B) \rar & \H^0(G,C) \ar[out=-20, in=160]{dll}
\\
& \H^1(G,A) \rar & \H^1(G,B) \rar & \H^1(G,C) \ar[out=-20, in=160]{dl}
\\
& & \cdots \ar[out=-20, in=160]{dl}
\\
& \H^n(G,A) \rar & \H^n(G,B) \rar & \H^n(G,C) \ar[out=-20, in=160]{dll}
\\
& \H^{n+1}(G,A) \rar & \H^{n+1}(G,B) \rar & \H^{n+1}(G,C) \rar & \cdots
	\end{tikzcd}
	$$
	Moreover, the functor $\H^n(G,-)$ is uniquely determined by
	these three properties.
\end{theorem}

We will not prove this theorem.  For proofs see
\cite[Atiyah-Wall]{cassels-frohlich} and
\cite[Ch.~7]{serre:localfields}. The properties of the theorem
uniquely determine group cohomology, so one should in theory be able
to use them to deduce anything that can be deduced about cohomology
groups.  Indeed, in practice one frequently proves results about
higher cohomology groups $\H^n(G,A)$ by writing down appropriate exact
sequences, using explicit knowledge of $\H^0$, and chasing diagrams.

\begin{remark}
	Alternatively, we could view the defining properties of the theorem
	as the definition of group cohomology, and could state a theorem
	that asserts that group cohomology exists.
\end{remark}

\begin{remark}
	For those familiar with commutative and homological algebra, we have
	$$
		\H^n(G,A) = \Ext^n_{\Z[G]}(\Z, A),
	$$
	where $\Z$ is the trivial $G$-module.
\end{remark}

\begin{remark}
	One can interpret $\H^2(G,A)$ as the group of equivalence classes of
	extensions of $G$ by $A$, where an extension is an exact sequence
	$$0\to A \to M \to G \to 1$$ such that the induced conjugation action
	of $G$ on $A$ is the given action of~$G$ on~$A$.
	(Note that $G$ acts by conjugation, as $A$ is a normal
	subgroup since it is the kernel of a homomorphism.)
\end{remark}

\subsection{Example Application of the Theorem}

For example, let's see what we get from the exact sequence
$$
	0 \to \Z \xrightarrow{m} \Z \to \Z/m\Z \to 0,
$$
where $m$ is a positive integer, and $\Z$ has the structure of
trivial~$G$ module.  By definition we have
$\H^0(G,\Z) = \Z$ and $\H^0(G,\Z/m\Z)=\Z/m\Z$.
The long exact sequence begins
$$
\begin{tikzcd}
	0 \rar & \Z \rar{m} & \Z \rar & \Z/m\Z \ar[out=-20, in=160]{dll}
	\\
	& \H^1(G,\Z) \rar{[m]} & \H^1(G,\Z) \rar & \H^1(G,\Z/m\Z) \ar[out=-20, in=160]{dll}
	\\
	& \H^2(G,\Z) \rar{[m]} & \H^2(G,\Z) \rar & \H^2(G,\Z/m\Z) \rar & \cdots
\end{tikzcd}
$$
From the first few terms of the sequence and the fact
that $\Z$ surjects onto $\Z/m\Z$, we see that
$[m]:\H^1(G,\Z) \to \H^1(G,\Z)$ is injective.
This is consistent with Exercise~\ref{ex:H1hom} above that
showed $\H^1(G,\Z) = 0$. Using this vanishing and the right side
of the exact sequence we obtain an isomorphism
$$
\H^1(G,\Z/m\Z) \isom \H^2(G,\Z)[m]
$$
where $\H^2(G,\Z)[m]$ is the kernel of the map
$[m]:\H^2(G,\Z) \to \H^2(G,\Z)$.
By Exercise~\ref{ex:H1hom}, when a group acts trivially the $\H^1$
is $\Hom$, so
\begin{equation}\label{eqn:h2}
	\H^2(G,\Z)[m] \isom \Hom(G,\Z/m\Z).
\end{equation}
One can prove that for any $n>0$ and any module~$A$ that the group
$\H^n(G,A)$ has exponent dividing $\#G$ (see Remark~\ref{rmk:cores}).
Thus (\ref{eqn:h2}) allows
us to understand $\H^2(G,\Z)$, and this comprehension arose
naturally from the properties in Theorem~\ref{thm:cohomology}
that determine the cohomology groups $\H^n$.



\section{Inflation and Restriction}

Suppose $H$ is a subgroup of a finite group~$G$ and $A$
is a $G$-module.

For each~$n\geq 0$, there is a natural map
$$
	\res_H : \H^n(G,A) \to \H^n(H,A)
$$
called \emph{restriction}. Elements of $\H^n(G,A)$ can be
viewed as classes of $n$-cocycles, which are certain maps
$G \times \cdots \times G \to A$. From this perspective $\res_H$
takes a map to its restriction $H \times \cdots \times H \to A$.
This is equivalent to precomposing with the natural inclusion
$H\times\cdots\times H \to G\times\cdots\times G$.

If~$H$ is a normal subgroup of~$G$, there is also an \emph{inflation} map
$$
	\inf_H: \H^n(G/H, A^H) \to \H^n(G,A),
$$
given by taking a cocycle $f : G/H \times \cdots \times G/H \to A^H$
and precomposing with the quotient map $G\to G/H$ to
obtain a cocycle for $G$.

\begin{exercise}
	Show that if $A$ is a $G$-module then
	$A^H$ is naturally a $G/H$-module for
	any normal subgroup $H$.
	Then give an example in which $G$ acts non-trivially on $A^H$
	but the only action of $G/H$ on $A$ is trivial.
	%Z/6Z acts non-trivially on Z/4Z but the only Z/3Z action is trivial
\end{exercise}

The following proposition will be useful when proving
the weak Mordell-Weil theorem (see Theorem~\ref{thm:weakMW}).
\begin{proposition}\label{prop:infres}
	Suppose $H$ is a normal subgroup of~$G$.
	Then there is an exact sequence
	$$
		0 \to \H^1(G/H, A^H)\xra{\inf_H}  \H^1(G,A)\xra{\res_H} \H^1(H,A).
	$$
\end{proposition}
\begin{proof}
Our proof follows \cite[pg.~117]{serre:localfields} closely.

We see that $\res\circ \inf = 0$ since on cocycles the composition is
defined by precomposing with $H\to G\to G/H$, which gives the trivial map.
It remains to prove that $\inf_H$ is injective and that the image of $\inf_H$
contains the kernel of $\res_H$.
\begin{enumerate}
\item {\em (That $\inf_H$ is injective):}
	Suppose $f:G/H\to A^H$ is a cocycle whose image in $\H^1(G,A)$
	is equivalent to~$0$ modulo coboundaries. Then there is an~$a\in A$
	such that $f(\sigma) = \sigma a - a$, where we identify~$f$ with
	the map $G\to A$ that is constant on the cosets of~$H$. But $f$
	depends only on the coset of $\sigma$ modulo~$H$, so
	$\sigma a - a = \sigma \tau a - a$ for all $\tau \in H$, i.e.,
	$\tau a = a$ (as we see by adding $a$ to both sides and multiplying
	by $\sigma^{-1}$). Thus $a\in A^H$, so $f$ is equivalent to~$0$ in
	$\H^1(G/H,A^H)$.

\item {\em (The image of $\inf_H$ contains the kernel of $\res_H$):}
	Suppose $f:G\to A$ is a cocycle whose
	restriction to $H$ is a coboundary, i.e., there is $a\in A$ such
	that $f(\tau) = \tau a - a$ for all $\tau \in H$.
	Subtracting the coboundary $g(\sigma) = \sigma a - a$ for $\sigma\in G$
	from~$f$, we may assume $f(\tau) = 0$ for all $\tau \in H$.
	Examing the equation $f(\sigma\tau) = f(\sigma) + \sigma f(\tau)$
	with $\tau\in H$ shows that $f$ is constant on the cosets of~$H$.
	Again using this formula, but with $\sigma\in H$ and $\tau\in G$, we see
	that
	$$
	  f(\tau) = f(\sigma \tau) = f(\sigma) + \sigma f(\tau) = \sigma f(\tau),
	$$
	so the image of~$f$ is contained in $A^H$.  Thus $f$ defines a cocycle
	$G/H \to A^H$,~i.e., is in the image of $\inf_H$.
\end{enumerate}
\end{proof}

\begin{example}
	The sequence of Proposition~\ref{prop:infres} need not be
	surjective on the right.  For example, suppose $H=A_3 \subset S_3$,
	and let $S_3$ act trivially on the group $\Z/3\Z$.
	Using the $\Hom$ interpretation of $\H^1$, we see
	that
	$\H^1(S_3/A_3, \Z/3\Z) = \H^1(S_3, \Z/3\Z) = 0$, but
	$\H^1(A_3, \Z/3\Z)$ has order~$3$.
	We can compute this example in \sage as follows.
\begin{sagecode}
\begin{sagecell}
S3 = SymmetricGroup(3); S3
\end{sagecell}
\begin{sageout}
Symmetric group of order 3! as a permutation group
\end{sageout}
\begin{sagecell}
S3.cohomology(1,3)
\end{sagecell}
\begin{sageout}
Trivial Abelian group
\end{sageout}
\begin{sagecell}
A3 = AlternatingGroup(3); A3
\end{sagecell}
\begin{sageout}
Alternating group of order 3!/2 as a permutation group
\end{sageout}
\begin{sagecell}
A3.cohomology(1,3)
\end{sagecell}
\begin{sageout}
Multiplicative Abelian group isomorphic to C3
\end{sageout}
\end{sagecode}
\end{example}


\begin{remark}
	One generalization of Proposition~\ref{prop:infres} is to
	a more complicated exact sequence involving the ``transgression map''
	tr:
	$$
		0 \to \H^1(G/H, A^H)\xra{\inf_H} \H^1(G,A)\xra{\res_H} \H^1(H,A)^{G/H}
		\xra{{\rm tr}}  \H^2(G/H,A^H) \to \H^2(G,A).
	$$
	Another generalization of Proposition~\ref{prop:infres}
	is that if $\H^m(H,A) = 0$ for $1\leq m < n$, then
	there is an exact sequence
	$$
		0 \to \H^n(G/H, A^H)\xra{\inf_H}  \H^n(G,A)\xra{\res_H} \H^n(H,A).
	$$
	For more information see \cite[Ch.~VII.6]{serre:localfields}.
\end{remark}

\begin{remark}\label{rmk:cores}
	If $H$ is a not-necessarily-normal subgroup of~$G$, there are also
	maps
	$$
		\cores_H: \H^n(H,A) \to \H^n(G,A)
	$$
	for each~$n$.  For $n=0$ this is the trace map
	$a\mapsto \sum_{\sigma \in G/H} \sigma a$, but the
	definition for $n\geq 1$ is more involved. One has
	$\cores_H \circ \res_H = [\#(G/H)]$. Taking $H=1$ this
	implies that for each $n\geq 1$ the group
	$\H^n(G,A)$ is annihilated by $[\#G]$.
\end{remark}

\section{Galois Cohomology}

Suppose $L/K$ is a finite Galois extension of fields
(recall that Galois here means is normal and separable),
and $A$ is a $\Gal(L/K)$-module.
Put
$$
	\H^n(L/K, A) = \H^n(\Gal(L/K), A).
$$

Following Section~\ref{sec:artin}, we can put a topology
on $\Gal(K^{\sep}/K)$ by taking as a basis of the origin,
subgroups of the form $\Gal(K^{\sep}/L)$ where $L/K$
is a finite Galois extension.

\begin{exercise}
	Let $H$ be a subgroup of $G = \Gal(K^{\sep}/K)$.
	Show that $H$ is open if and only if $H$ is closed
	and has finite index in $G$.

	\begin{hint}
		If $H$ is open then it contains a basis element $N$.
		By definition of the basis described above, $N$ is
		finite index in $G$. What does this say about
		the index of $H$ in $G$? What about the complement
		of $H$?
	\end{hint}
	% solution:
	% If H open then contains basis element N which is finite
	% index and open. Note all cosets of N are also open because
	% multiplication by elements is homeomorphism. It follows the complement
	% of N is finite union of open things, hence N is also closed.
	% This also shows G/N is discrete, so H is closed and finite index.
	%
	% If H is closed and finite index, then do the same trick as
	% above by considering cosets and multiplication maps.
\end{exercise}

\begin{definition}
	Let $A$ be a $\Gal(K^{\sep}/K)$-module. We say that $A$
	is a \emph{continuous} $\Gal(K^{\sep}/K)$-module if the map
	$\Gal(K^{\sep}/K)\times A \to A$
	(see Exercise~\ref{ex:equivalentdata}) is continuous when $A$
	has the discrete topology.
\end{definition}

\begin{exercise}
	Let $G = \Gal(K^{\sep}/K)$ and $A$ be a $G$-module.
	Show that $A$ is a continuous $G$-module
	if and only if the subgroup
	$G_a = \{\sigma \in G : \sigma(a) = a\}$ is open
	for every $a\in A$.
\end{exercise}

Now let $A$ be a continuous $\Gal(K^{\sep}/K)$-module. Let
$$
	A(L) = A^{\Gal(K^{\sep}/L)} = \{x \in A : \sigma(x) = x
	\text{ for all } \sigma \in\Gal(K^{\sep}/L)\}.
$$
and define
$$
	\H^n(K,A) = \varinjlim_{L/K} \H^n(L/K,A(L)),
$$
where the limit is taken over all finite Galois
extensions $L/K$.

It is not obvious that the groups $\H^n(K,A)$ are
actually cohomology groups, i.e., they satisfy the
conclusion of Theorem~\ref{thm:cohomology}. However
one can show they have analogous properties; see
\cite[Ch.~X.3]{serre:localfields} for references.

\begin{remark}
	Those familiar with algebraic geometry should
	compare the groups $\H^n(K,A)$ with the \v{C}ech
	cohomology groups on the \'{e}tale site over $\Spec K$.
	One can show that \v{C}ech cohomology
	agrees with the derived functor groups of
	$A\mapsto A^G$, see \cite[Ch.~10]{milne:etale}.
	Therefore $\H^n(K,A)$ do indeed define a cohomology
	theory.
\end{remark}

\begin{example}
The following are examples of continuous $\Gal(\Qbar/\Q)$-modules:
$$
	\Qbar,
	\quad \Qbar^*,
	\quad \Zbar,
	\quad \Zbar^*,
	\quad E(\Qbar),
	\quad E(\Qbar)[n],
	\quad {\rm Tate}_{\ell}(E),
$$
where $E$ is an elliptic curve over~$\Q$. Can you identify the
action for each module $A$? What about $A(L)$ for any finite
Galois extension $L/\Q$. It is important to notice that
$\Qbar^*(L) = L^*$.
\end{example}

\begin{theorem}[Hilbert 90]\label{thm:h90}
	We have $\H^1(K,\Kbar^*) = 0$.
\end{theorem}
\begin{proof}
	Our proof follows \cite[pg.~150]{serre:localfields} closely.

	Because $\H^1(K,\Kbar^*) = \varinjlim_{L/K} \H^1(L/K,L^*)$
	It suffices to prove $\H^1(L/K,L^*) = 0$ for every finite
	Galois extension $L/K$.
	Let $G = \Gal(L/K)$ and $f$ be a $1$-cocycle so that $f:G \to L^*$
	such that $f(\sigma\tau) = f(\sigma)\cdot\sigma(f(\tau))$. Here
	``~$\cdot$~'' represents multiplication in $L^*$.
	A standard fact from Galois theory is that the elements of
	$G$ are $L$ linearly independent. Hence we can find
	some $c\in L$ such that
	$$
	b = \sum_{\tau\in G} f(\tau)\cdot\tau(c) \neq 0.
	$$
	Now apply $\sigma$ to both sides to get
	\begin{align*}
		\sigma(b)
		&=
		\sum_{\tau\in G} \sigma(f(\tau)) \cdot \sigma\tau(c)
		\\
		&=
		\sum_{\tau\in G} f(\sigma)^{-1} \cdot f(\sigma\tau)
		\cdot \sigma\tau(c)
		\\
		&=
		f(\sigma)^{-1} \cdot \sum_{\tau\in G} f(\sigma\tau)
		\cdot (\sigma\tau)(c)
		\\
		&=
		f(\sigma)^{-1}\cdot b.
	\end{align*}
	This shows $f$ is a coboundary. Specifically, it shows
	$f = f_{b^{-1}}$ in the notation we used to define
	coboundaries above.
\end{proof}

%%% Local Variables:
%%% mode: latex
%%% TeX-master: "ant"
%%% End: