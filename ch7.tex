%%%%%%%%%%%%%%%%%%%%%%%%%%%%%%%%%%%%%%%%%%%%%%%%%%%%%%%%%%%%%%%%%%%%%%%%%%
%% CLASSGROUP
%%%%%%%%%%%%%%%%%%%%%%%%%%%%%%%%%%%%%%%%%%%%%%%%%%%%%%%%%%%%%%%%%%%%%%%%%%

\chapter{Finiteness of the Class Group}\label{ch:classgroup}
Frequently $\sO_K$ is not a principal ideal domain.  This chapter is
about a way to understand how badly $\sO_K$ fails to be a principal
ideal domain.  The class group of $\sO_K$ measures this failure.  As
one sees in a course on Class Field Theory, the class group and its
generalizations also yield deep insight into the
extensions of~$K$ that are Galois with abelian Galois group.

In Section~\ref{sec:theclassgroup}, we define the class group and
state the main theorem of this chapter.  We then illustrate the
implications of this theorem in detail for the field $\QQ(\sqrt{10})$,
proving that it has class group of order $2$. Next, we prove several
geometric lemmas, building very heavily on ours results from
Chapter~\ref{discnorm}.  Finally, we close the section by giving a
complete proof of finiteness of the class group, but leave an explicit
upper bound as an exercise in calculus.  In Section~\ref{sec:cn1} we
very briefly discuss how often number fields have class number 1.
Finally, in Section~\ref{sec:comcg} we further discuss how to compute
class groups, though nothing we do in this book begins to approach the
state of the art regarding such computations -- for that, see Cohen's
books.

\section{The Class Group}\label{sec:theclassgroup}

\begin{definition}[Class Group]
	Let $\sO_K$ be the ring of integers of a number field~$K$.  The
	\defn{class group} $C_K$ of~$K$ is the group of fractional ideals
	modulo the subgroup of principal fractional ideals $(a)$, for $a\in K$.
\end{definition}

Note that if we let $\Div(\sO_K)$ denote the group of  fractional
ideals, then we have an exact sequence
\[
	0 \to \sO_K^* \to K^* \to \Div(\sO_K) \to C_K \to 0.
\]

That the class group $C_K$ is finite follows from the first part of
the following theorem and that there are only finitely many
ideals of norm less than a given integer (Proposition~\ref{prop:finitewithnorm}).
\begin{theorem}[Finiteness of the Class Group]\label{thm:finiteclassgrp}
	\ithm{finiteness of class group}
	Let $K$ be a number field.  There is a constant $C_{r,s}$ that
	depends only on the number $r$, $s$ of real and pairs
	of complex conjugate embeddings of~$K$, respectively, such that
	every ideal class of $\sO_K$ contains an integral ideal
	of norm at most $C_{r,s}\sqrt{|d_K|}$, where $d_K=\Disc(\sO_K)$.
	Thus by Proposition~\ref{prop:finitewithnorm}
	the class group $C_K$ of~$K$ is finite. In fact, one can take
	\[
		C_{r,s} = \left(\frac{4}{\pi}\right)^s\frac{n!}{n^n}.
	\]
\end{theorem}
The explicit bound in the theorem
\[
	M_K = \left(\frac{4}{\pi}\right)^s\frac{n!}{n^n} \cdot \sqrt{|d_K|}
\]
is called the \defn{Minkowski bound}.
There are other better bounds, but they depend on unproven conjectures
\cite{bach1990explicit}.

The following two examples illustrate how to apply
Theorem~\ref{thm:finiteclassgrp} to compute $C_K$ in simple cases.
\begin{example}
	Let $K=\QQ[i]$.  Then $n=2$, $s=1$, and $|d_K|=4$, so the Minkowski bound is
	\[
		\sqrt{4} \cdot \left(\frac{4}{\pi}\right)^1 \frac{2!}{2^2}
		= \frac{4}{\pi} < 2.
	\]
	Thus every fractional ideal is equivalent to an ideal of norm~$1$.
	Since $(1)$ is the only ideal of norm $1$, every ideal is principal,
	so $C_K$ is trivial.
\end{example}

\begin{example}
	Let $K=\QQ(\sqrt{10})$. We have $\sO_K = \ZZ[\sqrt{10}]$,
	so $n=2$, $s=0$, $|d_K| = 40$, and the Minkowski bound is
	\[
		\sqrt{40}\cdot \left(\frac{4}{\pi}\right)^0 \cdot \frac{2!}{2^2}
		= 2\cdot \sqrt{10} \cdot \frac{1}{2} = \sqrt{10} = 3.162277\dots.
	\]
	We compute the Minkowski bound in {\Sage} as follows:
	\begin{sagecode}
		\begin{sagecell}
			K = QQ[sqrt(10)]; K
		\end{sagecell}
		\begin{sageout}
			Number Field in sqrt10 with defining polynomial x^2 - 10
		\end{sageout}
		\begin{sagecell}
			B = K.minkowski_bound(); B
		\end{sagecell}
		\begin{sageout}
			sqrt(10)
		\end{sageout}
		\begin{sagecell}
			B.n()
		\end{sagecell}
		\begin{sageout}
			3.16227766016838
		\end{sageout}
	\end{sagecode}
	Theorem~\ref{thm:finiteclassgrp} implies that every ideal class has a
	representative that is an integral ideal of norm~$1$,~$2$, or~$3$.
	The ideal $2\sO_K$ is ramified in $\sO_K$, so
	$$
	2\sO_K = (2,\sqrt{10})^2.
	$$
	If $(2,\sqrt{10})$ were principal, say $(\alpha)$, then
	$\alpha=a+b\sqrt{10}$ would have norm $\pm 2$.
	Then the equation
	\begin{equation}\label{eqn:norm10}
	x^2 - 10y^2 = \pm 2,
	\end{equation}
	would have an integer solution.  But the squares mod~$5$ are
	$0,\pm 1$, so (\ref{eqn:norm10}) has no solutions.
	Thus $(2,\sqrt{10})$ defines a nontrivial element of the class group,
	and it has order~$2$ since its square is the principal ideal $2\sO_K$.
	Thus $2\mid \#C_K$.
	
	To find the integral ideals of norm $3$, we
	factor $x^2-10$ modulo~$3$, and see that
	$$
	3\sO_K  = (3,2+\sqrt{10}) \cdot (3,4+\sqrt{10}).
	$$
	If either of the prime divisors of $3\sO_K$ were principal,
	then the equation $x^2-10y^2 = \pm 3$ would have an integer
	solution.  Since it does not have one mod~$5$, the prime divisors
	of $3\sO_K$ are both nontrivial elements of the class
	group. Let
	$$
	\alpha = \frac{4+\sqrt{10}}{2+\sqrt{10}} = \frac{1}{3}\cdot (1+\sqrt{10}).
	$$
	Then
	$$
	(3,2+\sqrt{10})\cdot (\alpha) =  (3\alpha, 4+\sqrt{10})
	=  (1+\sqrt{10}, 4+\sqrt{10})
	=  (3, 4+\sqrt{10}),
	$$
	so the classes over~$3$ are equal.
	
	In summary, we now know that every element of $C_K$ is equivalent to one of
	$$
	(1),\quad (2,\sqrt{10}), \quad \text{ or } \quad (3,2+\sqrt{10}).
	$$
	Thus the class group is a group of order at most $3$ that contains an
	element of order~$2$.  Thus it must have order~$2$.  We verify this in
	{\Sage} below, where we also check that $(3, 2+\sqrt{10})$ generates the
	class group.
	\begin{sagecode}
		\begin{sagecell}
			K.<sqrt10> = QQ[sqrt(10)]; K
		\end{sagecell}
		\begin{sageout}
			Number Field in sqrt10 with defining polynomial x^2 - 10
		\end{sageout}
		\begin{sagecell}
			G = K.class_group(); G
		\end{sagecell}
		\begin{sageout}
			Class group of order 2 with structure C2 of Number Field ...
		\end{sageout}
		\begin{sagecell}
			G.0
		\end{sagecell}
		\begin{sageout}
			Fractional ideal class (3, sqrt10 + 1)
		\end{sageout}
		\begin{sagecell}
			G.0^2
		\end{sagecell}
		\begin{sageout}
			Trivial principal fractional ideal class
		\end{sageout}
		\begin{sagecell}
			G.0 == G( (3, 2 + sqrt10) )
		\end{sagecell}
		\begin{sageout}
			True
		\end{sageout}
	\end{sagecode}
\end{example}

Before proving Theorem~\ref{thm:finiteclassgrp}, we prove a few
lemmas.  The strategy of the proof is to start with any nonzero
ideal~$I$, and prove that there is some nonzero $a\in K$ having very
small norm, such that $aI$ is an integral ideal. Then
$\Norm(aI)=\Norm_{K/\QQ}(a)\Norm(I)$ will be small, since
$\Norm_{K/\QQ}(a)$ is small.  The trick is to determine precisely
how small an $a$ we can choose subject to the condition that
$aI$ is an integral ideal, i.e., that $a\in I^{-1}$.

Let $S$ be a subset of $V=\RR^n$.  Then $S$ is \defn{convex} if
whenever $x,y\in S$ then the line connecting $x$ and $y$ lies entirely
in $S$.  We say that $S$ is \defn{symmetric about the origin} if
whenever $x\in S$ then $-x\in S$ also.  If $L$ is a lattice in the
real vector space $V=\RR^n$, then the \defn{volume} of $V/L$ is the
volume of the compact real manifold $V/L$, which is the same thing as
the absolute value of the determinant of any matrix whose rows form a
basis for~$L$.
\begin{lemma}[Blichfeld]\label{lem:blichfeld}\ilem{Blichfeld}
	Let $L$ be a lattice in $V=\RR^n$, and let $S$ be a
	bounded closed convex subset of $V$ that is symmetric about the
	origin.  If
	$\Vol(S)\geq 2^n \Vol(V/L),$
	then~$S$ contains a nonzero element of $L$.
\end{lemma}
\begin{proof}
	First assume that $\Vol(S)>2^n \Vol(V/L)$.
	If the map $\pi: \frac{1}{2}S \to V/L$ is injective, then
	$$
	\frac{1}{2^n}\Vol(S) = \Vol\left(\frac{1}{2} S\right)\leq \Vol(V/L),
	$$
	a contradiction.  Thus $\pi$ is not injective, so there
	exist $P_1\neq P_2\in \frac{1}{2}S$ such that $P_1-P_2\in L$.
	Because $S$ is symmetric about the origin, $-P_2\in \frac{1}{2}S$.
	By convexity, the average $\frac{1}{2}(P_1-P_2)$ of $P_1$ and $-P_2$
	is also in $\frac{1}{2}S$.  Thus $0 \neq P_1 - P_2 \in S \cap L$,
	as claimed.
	
	Next assume that $\Vol(S) = 2^n\cdot \Vol(V/L)$.  Then for all
	$\varepsilon > 0$ there is $0\neq Q_\varepsilon \in L \cap (1 + \varepsilon) S$,
	since $\Vol((1 + \varepsilon)S) > \Vol(S) = 2^n \cdot \Vol(V/L)$.
	If $\varepsilon < 1$ then the $Q_\varepsilon$ are all in $L \cap 2 S$,
	which is finite since $2S$ is bounded and $L$ is discrete.
	Hence there exists nonzero $Q = Q_\varepsilon \in L\cap (1 + \varepsilon) S$
	for arbitrarily small $\varepsilon$.
	Since $S$ is closed, $Q \in L \cap S$.
\end{proof}

\begin{lemma}\label{lem:latticevolchange}\ilem{lattices and volumes}
	If $L_1$ and $L_2$ are lattices in $V$, then
	\[
		\Vol(V/L_2) = \Vol(V/L_1) \cdot [L_1:L_2].
	\]
\end{lemma}
\begin{proof}
	Let $A$ be an automorphism of~$V$ such that $A(L_1)=L_2$.  Then~$A$
	defines an isomorphism of real manifolds $V/L_1\to V/L_2$ that changes
	volume by a factor of $\left|\det(A)\right|=[L_1:L_2]$.
	The claimed formula then follows,
	since $[L_1:L_2] = \left|\det(A)\right|$, by definition.
\end{proof}

Fix a number field $K$ with ring of integers $\sO_K$.
Let $\sigma_1,\dots, \sigma_r$ be the real embeddings
of $K$ and $\sigma_{r+1},\dots, \sigma_{r+s}$ be half
the complex embeddings of $K$, with one representative of
each pair of complex conjugate embeddings.
Let $\sigma:K \to V=\RR^n$ be the embedding
\begin{align*}
\sigma(x) = \big(&\sigma_1(x), \sigma_2(x),\dots, \sigma_r(x), \\
&\quad \Real(\sigma_{r+1}(x)), \dots, \Real(\sigma_{r+s}(x)),
\Imag(\sigma_{r+1}(x)), \dots, \Imag(\sigma_{r+s}(x))\big),
\end{align*}
\begin{warning}
	This $\sigma$ is {\em not} the same as the one
	at the beginning of Section~\ref{sec:disc} if $s>0$.
\end{warning}
\begin{lemma}\label{lem:volok}\ilem{volume of rings of integers}
	Let $\sigma$ be the map described above. Then
	$$
	\Vol(V/\sigma(\sO_K)) = 2^{-s} \sqrt{\left|d_K\right|}.
	$$
\end{lemma}
\begin{proof}
	Let $L=\sigma(\sO_K)$.
	From a basis $w_1,\dots, w_n$ for $\sO_K$ we obtain a matrix $A$
	whose $i$th row is
	$$
	(\sigma_1(w_i), \cdots, \sigma_r(w_i),
	\Real(\sigma_{r+1}(w_i)),\dots, \Real(\sigma_{r+s}(w_i)),
	\Imag(\sigma_{r+1}(w_i)),\dots, \Imag(\sigma_{r+s}(w_i)))
	$$
	and whose determinant has absolute value equal to the volume
	of $V/L$.  By doing the following three column operations,
	we obtain a matrix whose rows are exactly the images of
	the $w_i$ under {\em all} embeddings of $K$ into $\CC$, which
	is the matrix that came up when we defined
	$d_K=\Disc(\sO_K)$ in Section~\ref{sec:disc}.
	\begin{enumerate}
		\item
		Add $i=\sqrt{-1}$ times each column with entries $\Imag(\sigma_{r+j}(w_i))$
		to the column with entries $\Real(\sigma_{r+j}(w_i))$.
		\item
		Multiply all columns with entries $\Imag(\sigma_{r+j}(w_i))$
		by $-2i$, thus changing the determinant by $(-2i)^s$.
		\item
		Add each column that now has entries
		$\Real(\sigma_{r+j}(w_i))+i\Imag(\sigma_{r+j}(w_i))$
		to the column with entries $-2i\Imag(\sigma_{r+j}(w_i))$
		to obtain columns $\Real(\sigma_{r+j}(w_i))-i\Imag(\sigma_{r+j}(w_i))$.
	\end{enumerate}
	Recalling the definition of discriminant, we see that if~$B$
	is the matrix constructed by doing the above three
	operations to $A$, then $\left|\det(B)^2\right| = \left|d_K\right|$.
	Thus
	$$
	\Vol(V/L)
	= \left|\det(A)\right|
	= \left|(-2i)^{-s}\cdot \det(B)\right|
	= 2^{-s}\sqrt{\left|d_K\right|}.
	$$
\end{proof}

\begin{lemma}\label{lem:volfracideal}\ilem{fractional ideal is lattice}
	If $I$ is a fractional $\sO_K$-ideal, then $\sigma(I)$ is
	a lattice in~$V$ and
	$$
	\Vol(V/\sigma(I)) = 2^{-s}\sqrt{|d_K|}\cdot \Norm(I).
	$$
\end{lemma}
\begin{proof}
	Since $\sigma(\sO_K)$ has rank $n$ as an abelian group, and
	Lemma~\ref{lem:volok} implies that $\sigma(\sO_K)$ also spans $V$,
	it follows that $\sigma(\sO_K)$ is a lattice in $V$.
	For some nonzero integer $m$ we have
	$m\sO_K \subset I \subset \frac{1}{m}\sO_K$,
	so $\sigma(I)$ is also a lattice in~$V$.
	To prove the displayed volume
	formula, combine Lemmas
	\ref{lem:latticevolchange} and \ref{lem:volok} to get
	$$
	\Vol(V/\sigma(I))
	= \Vol(V/\sigma(\sO_K))\cdot[\sO_K:I]
	= 2^{-s}\sqrt{|d_K|}\Norm(I).
	$$
\end{proof}


\begin{proof}[Proof of Theorem~\ref{thm:finiteclassgrp}]
	Let $K$ be a number field with ring of integers $\sO_K$,
	let $\sigma:K \hookrightarrow V \isom \RR^n$ be as above,
	and let $f:V\to \RR$ be the function defined by
	$$
	f(x_1,\dots, x_n)
	= \left|
	x_1 \cdots x_r\cdot (x_{r+1}^2 + x_{(r+1)+s}^2)\cdots (x_{r+s}^2 + x_n^2)
	\right|.
	$$
	Notice that if $x\in K$ then $f(\sigma(x)) = |\Norm_{K/\QQ}(x)|$,
	and for any $a\in \RR$,
	$$
	f(ax_1, \dots,  ax_n) = |a|^n f(x_1,\dots, x_n).
	$$
	
	Let $S\subset V$ be any fixed choice of closed, bounded, convex, subset with
	positive volume that is symmetric with respect to the origin.
	Since~$S$ is closed and bounded,
	$$
	M = \max\{f(x) : x \in S\}
	$$
	exists.
	
	Suppose~$I$ is any  fractional ideal of $\sO_K$.  Our goal
	is to prove that there is an integral ideal $aI$ with small norm. We
	will do this by finding an appropriate $a\in I^{-1}$.
	By Lemma~\ref{lem:volfracideal},
	$$
	c = \Vol(V/\sigma(I^{-1}))
	= 2^{-s}\sqrt{|d_K|}\cdot \Norm(I)^{-1}
	= \frac{2^{-s} \sqrt{|d_K|}}{\Norm(I)}.
	$$
	Let $\lambda = 2\cdot\left(\frac{c}{v}\right)^{1/n}$, where $v=\Vol(S)$.
	Then
	$$
	\Vol(\lambda{} S) = \lambda^n \Vol(S)
	= 2^n\cdot \frac{c}{v} \cdot v
	= 2^n\cdot c
	= 2^n \Vol(V/\sigma(I^{-1})),
	$$
	so by Lemma~\ref{lem:blichfeld} there exists
	$0\neq b\in \sigma(I^{-1}) \cap \lambda S$.
	Let $a \in I^{-1}$ be such that $\sigma(a)=b$.
	Since~$M$ is the largest norm of an element of~$S$, the largest norm
	of an element of $\sigma(I^{-1})\cap  \lambda{}S$ is at most $\lambda^n M$,
	so
	$$
	\left|\Norm_{K/\QQ}(a)\right| \leq \lambda^n M.
	$$
	Since $a\in I^{-1}$, we have $aI \subset \sO_K$, so
	$aI$ is an integral ideal of $\sO_K$ that is equivalent to $I$, and
	\begin{align*}
	\Norm(aI) &= \left|\Norm_{K/\QQ}(a)\right|\cdot \Norm(I) \\
	&\leq \lambda^n M\cdot \Norm(I) \\
	&\leq 2^n \frac{c}{v} M \cdot \Norm(I) \\
	&= 2^n\cdot 2^{-s} \sqrt{|d_K|} \cdot M \cdot v^{-1} \\
	&= 2^{r+s} \sqrt{|d_K|} \cdot M \cdot v^{-1}.
	\end{align*}
	Notice that the right hand side is independent of $I$.  It
	depends only on $r$, $s$, $|d_K|$, and our choice of~$S$.
	This completes the proof of the theorem, except for
	the assertion that $S$ can be chosen to give the claim
	at the end of the theorem which is shown in Exercise~\ref{ex:canchooseSright}.
\end{proof}

\begin{exercise}\label{ex:canchooseSright}
	Prove that the quantity $C_{r,s}$ in Theorem~\ref{thm:finiteclassgrp}
	can be taken to be $\left(\frac{4}{\pi}\right)^{s} \frac{n!}{n^n}$,
	as follows (adapted from \cite[pg.~19]{sd:brief}):
	Let $S$ be the set of elements $(x_1,\dots, x_{n})\in\RR^n$ such that
	$$
	|x_1| + \cdots |x_{r}| +
	2 \sum_{v=r+1}^{r+s} \sqrt{x_v^2 + x_{v+s}^2} \leq 1.
	$$
	\begin{enumerate}
		\item Prove that $S$ is convex and that $M=n^{-n}$, where 
		$$
		M = \max\left\{
		\left|x_1 \cdots x_r
		\cdot (x_{r+1}^2 + x_{(r+1)+s}^2) \cdots (x_{r+s}^2 + x_n^2)\right|
		\colon (x_1,\dots, x_n) \in S
		\right\}.
		$$
		
		\begin{hint}
			For convexity, use the triangle inequality and that for
			$0\leq \lambda \leq 1$, we have
			\begin{align*}
			\lambda\sqrt{x_1^2 + y_1^2} &+ (1-\lambda)\sqrt{x_2^2+y_2^2} \\
			&\geq\sqrt{(\lambda x_1 + (1 - \lambda)x_2)^2
				+ (\lambda y_1 + (1 - \lambda)y_2)^2}
			\end{align*}
			for $0\leq \lambda \leq 1$.  In polar coordinates this last inequality
			is 
			$$
			\lambda r_1 + (1-\lambda)r_2 \geq  \sqrt{
				\lambda^2 r_1^2
				+ 2\lambda(1 - \lambda) r_1 r_2 \cos(\theta_1 - \theta_2)
				+ (1 - \lambda)^2 r_2^2
			},
			$$
			which is trivial.  That $M\leq n^{-n}$ follows from the inequality
			between the arithmetic and geometric means.
		\end{hint}
		\item Transforming pairs $x_v, x_{v+s}$ from Cartesian to polar coordinates,
		show also that $\Vol(S) = 2^{r}(2\pi)^s D_{r,s}(1)$, where
		$$
		D_{\ell,m}(t) = \int \cdots \int_{\mathcal{R}_{\ell,m}(t)}
		y_1 \cdots y_m dx_1 \cdots dx_{\ell} dy_1 \cdots dy_m
		$$
		and 
		$\mathcal{R}_{\ell,m}(t)$ is given by $x_{\rho}\geq 0$
		($1\leq \rho\leq \ell$), $y_{\rho}\geq 0$ ($1\leq \rho\leq m$) and 
		$$
		x_1 + \cdots + x_{\ell} + 2(y_1+\cdots +y_m) \leq t.
		$$
		\item Prove that
		$$
		D_{\ell,m}(t) = \int_{0}^t D_{\ell-1,m}(t-x)dx
		= \int_{0}^{t/2} D_{\ell,m-1}(t-2y)y dy
		$$
		and deduce by induction that 
		$$
		D_{\ell,m}(t) = \frac{4^{-m}t^{\ell+2m}}{(\ell+2m)!}
		$$
	\end{enumerate}
\end{exercise}

\begin{corollary}\icor{discrimant of number field $>1$}\label{co:disc>1}
	Suppose that $K\neq \QQ$ is a number field.  Then $|d_K|>1$.
\end{corollary}
\begin{proof}
	Applying Theorem~\ref{thm:finiteclassgrp} to the unit ideal,
	we get the bound
	$$
	1\leq \sqrt{|d_K|}\cdot \left(\frac{4}{\pi}\right)^s\frac{n!}{n^n}.
	$$
	Thus
	$$
	\sqrt{|d_K|}
	\geq
	\left(\frac{\pi}{4}\right)^s\frac{n^n}{n!},
	$$
	and the right hand quantity is strictly bigger than $1$ for
	any $s\leq n/2$ and any $n>1$, see Exercise~\ref{ex:basicKbound}.
\end{proof}

\begin{exercise}\label{ex:basicKbound}
	Prove the statement at the end of the proof for Corollary~\ref{co:disc>1},
	i.e. suppose $n>1$ and $s\leq \frac{n}{2}$ as above. Show that
	$\left(\frac{\pi}{4}\right)^s\frac{n^n}{n!} > 1.$
\end{exercise}

A prime $p$ ramifies in $\sO_K$ if and only if $d\mid d_K$,
so the corollary implies that every nontrivial extension of $\QQ$
is ramified at some prime.

\begin{exercise}
	For each of the following three fields, determining if there is
	an order of discriminant $20$ contained in its ring of integers:
	\begin{enumerate}
		\item $K = \QQ(\sqrt{5})$
		\item $K = \QQ(\sqrt[3]{2})$
		\item $K = $ any extension of $\QQ$ of degree $2017$
	\end{enumerate}
	
	\begin{hint}
		Apply the exact form of our theorem about finiteness of class groups
		to the unit ideal to show that the discriminant of a degree $2017$
		field must be large.
	\end{hint}
\end{exercise}

\section{Class Number 1}\label{sec:cn1}

The fields of class number 1 are exactly the fields for
which $\sO_K$ is a principal ideal domain.  How many such
number fields are there?   We still don't know.
\begin{conjecture}
	There are infinitely many number fields~$K$ such that the class
	group of~$K$ has order~$1$.
\end{conjecture}
For example, if we consider real quadratic fields $K=\QQ(\sqrt{d})$,
with $d$ positive and square free, many class numbers are probably $1$,
as suggested by the {\Sage} output below.
It looks like 1's will keep appearing infinitely often, and indeed
Cohen and Lenstra conjecture that they do
(\cite{cohen-lenstra:heuristics}).\footnote{Specifically, Cohen and
	Lenstra conjecture that $75.446\dots\%$ of real quadratic fields
	with prime discriminant have class number $1$.}
\begin{sagecode}
	\begin{sagecell}
		for d in [2..1000]:
		if is_fundamental_discriminant(d):
		h = QuadraticField(d, 'a').class_number()
		if h == 1:
		print d
	\end{sagecell}
	\begin{sageout}
		5 8 12 13 17 21 24 28 29 33 37 41 44 53 56 57 61 69
		73 76 77 88 89 92 93 97 101 109 113 124 129 133 137
		141 149 152 157 161 172 173 177 181 184 188 193 197
		201 209 213 217 233 236 237 241 248 249 253 268 269
		277 281 284 293 301 309 313 317 329 332 337 341 344
		349 353 373 376 381 389 393 397 409 412 413 417 421
		428 433 437 449 453 457 461 472 489 497 501 508 509
		517 521 524 536 537 541 553 556 557 569 573 581 589
		593 597 601 604 613 617 632 633 641 649 652 653 661
		664 668 669 673 677 681 701 709 713 716 717 721 737
		749 753 757 764 769 773 781 789 796 797 809 813 821
		824 829 844 849 853 856 857 869 877 881 889 893 908
		913 917 921 929 933 937 941 953 956 973 977 989 997
	\end{sageout}
\end{sagecode}
In contrast, if we look at class numbers of quadratic imaginary fields,
only a few at the beginning have class number~$1$.
\begin{sagecode}
	\begin{sagecell}
		for d in [-1,-2..-1000]:
		if is_fundamental_discriminant(d):
		h = QuadraticField(d, 'a').class_number()
		if h == 1:
		print d
	\end{sagecell}
	\begin{sageout}
		-3 -4 -7 -8 -11 -19 -43 -67 -163
	\end{sageout}
\end{sagecode}
It is a theorem that was proved independently and in different ways by
Heegner, Stark, and Baker that the above list of $9$ fields is the
complete list with class number~$1$.  More generally, it is possible,
using deep work of Gross, Zagier, and Goldfeld involving zeta
functions and elliptic curves, to enumerate all quadratic number
fields with a given class number (Mark Watkins has done very
substantial work in this direction).

% The function in PARI for computing the order of the class group of a
% quadratic field in PARI is called {\tt qfbclassno}.
% \begin{verbatim}
% ?qfbclassno
%  qfbclassno(x,{flag=0}): class number of discriminant x using
%  Shanks's method by default. If (optional) flag is set to 1,
%  use Euler products.
% ? for(d=2,1000, if(isfundamental(d), h=qfbclassno(d);if(h==1,print1(d,", "))))
% 5, 8, 12, 13, 17, 21, 24, ... 977, 989, 997,

% ? for(d=-1000,-1,if(isfundamental(d), h=qfbclassno(d);if(h==1,print1(d,", "))))
% -163, -67, -43, -19, -11, -8, -7, -4, -3
% \end{verbatim}
% PARI does the above class number computations {\em vastly}
% faster than MAGMA.   However, note the following ominous warning
% in the PARI manual, which has been there in some form since 1997:
% \begin{quote}
% {\bf Important warning.} For $D<0$, this function may give incorrect
% results when the class group has a low exponent (has many cyclic factors),
% because implementing Shank's method in full generality slows it down
% immensely.  It is therefore strongly recommended to double-check results
% using either the version with {\em flag=1}, the fucntion {\tt qfbhclassno(-D)}
% or the function {\tt quadclassunit}.
% \end{quote}
% %The documentation used to say that Shank's method was not implemented
% %in full generality because ``the authors were too lazy''.


\section{More About Computing Class Groups}\label{sec:comcg}

If $\p$ is a prime of $\sO_K$, then the intersection $\p \cap \ZZ = p\ZZ$ is
a prime ideal of $\ZZ$.  We say that $\p$ \defn{lies over} $p\in\ZZ$.
Note $\p$ lies over $p\in\ZZ$ if and only if $\p$ is one of the prime
factors in the factorization of the ideal $p\sO_K$.  Geometrically,
$\p$ is a point of $\Spec(\sO_K)$ that lies over the point $p\ZZ$ of
$\Spec(\ZZ)$ under the map induced by the inclusion $\ZZ \hookrightarrow \sO_K$
as described in Section~\ref{sec:geom_intuition}.


\begin{lemma}\ilem{class group generated by bounded primes}
	Let $K$ be a number field with ring of integers $\sO_K$.  Then the
	class group $\Cl(K)$ is generated by the prime ideals $\p$ of $\sO_K$
	lying over primes $p\in\ZZ$ with $p\leq B_K = \sqrt{|d_K|}\cdot
	\left(\frac{4}{\pi}\right)^s\cdot \frac{n!}{n^n}$,
	where $s$ is the number of complex conjugate pairs of embeddings
	$K \hookrightarrow \CC$.
\end{lemma}
\begin{proof}
	Theorem~\ref{thm:finiteclassgrp}
	asserts that every ideal class in $\Cl(K)$ is represented by
	an ideal~$I$ with $\Norm(I)\leq B_K$.  Write $I=\prod_{i=1}^m
	\p_i^{e_i}$, with each $e_i\geq 1$.  Then by multiplicativity of the
	norm, each $\p_i$ also satisfies $\Norm(\p_i)\leq B_K$.  If $\p_i \cap
	\ZZ = p\ZZ$, then $p\mid \Norm(\p_i)$, since $p$ is the residue
	characteristic of $\sO_K/\p$, so $p\leq B_K$. Thus~$I$ is a product of
	primes~$\p$ that satisfies the norm bound of the lemma.
\end{proof}

This is a sketch of how to compute $\Cl(K)$:
\begin{enumerate}
	\item Use the algorithms of Chapter~\ref{ch:factoring_primes} to list all
	prime ideals $\p$ of $\sO_K$ that appear in the factorization
	of a prime $p\in \ZZ$ with $p\leq B_K$.
	\item Find the group generated  by the ideal
	classes $[\p]$, where the $\p$ are the prime ideals
	found in step 1.  (In general, this step can become
	fairly complicated.)
\end{enumerate}
The following three examples illustrate computation of $\Cl(K)$
for $K=\QQ(i), \QQ(\sqrt{5})$ and $\QQ(\sqrt{-6})$.
\begin{example}
	We compute the class group of $K=\QQ(i)$.  We have
	$$
	n = 2, \quad r=0, \quad s=1, \quad d_K = -4,
	$$
	so
	$$
	B_K = \sqrt{4}\cdot \left(\frac{4}{\pi}\right)^1
	\cdot\left(\frac{2!}{2^2}\right) = \frac{8}{\pi} <3.
	$$
	Thus $\Cl(K)$ is generated by the prime divisors
	of $2$.  We have
	$$
	2\sO_K = (1+i)^2,
	$$
	so $\Cl(K)$ is generated by the principal prime
	ideal $\p=(1+i)$. Thus $\Cl(K)=0$ is trivial.
\end{example}

\begin{example}
	We compute the class group of $K=\QQ(\sqrt{5})$.
	We have
	$$
	n = 2, \quad r = 2, \quad s=0, \quad d_K = 5,
	$$
	so
	$$
	B = \sqrt{5}\cdot \left(\frac{4}{\pi}\right)^0
	\cdot \left(\frac{2!}{2^2}\right) < 3.
	$$
	Thus $\Cl(K)$ is generated by the primes that divide $2$.
	We have $\sO_K=\ZZ[\gamma]$, where $\gamma=\frac{1+\sqrt{5}}{2}$
	satisfies $x^2-x-1$.   The polynomial $x^2-x-1$ is irreducible
	mod $2$, so $2\sO_K$ is prime.  Since it is principal, we see
	that $\Cl(K)=1$ is trivial.
\end{example}

\begin{example}
	In this example, we compute the class group of $K=\QQ(\sqrt{-6})$.
	We have
	$$
	n = 2, \quad r=0, \quad s=1, \quad d_K = -24,
	$$
	so
	$$
	B = \sqrt{24} \cdot \frac{4}{\pi} \cdot
	\left(\frac{2!}{2^2}\right)\sim 3.1.
	$$
	Thus $\Cl(K)$ is generated by the prime ideals lying over $2$ and $3$.
	We have $\sO_K=\ZZ[\sqrt{-6}]$, and $\sqrt{-6}$ satisfies $x^2+6=0$.
	Factoring $x^2+6$ modulo $2$ and $3$ we see that the class group
	is generated by the prime ideals
	$$
	\p_2 = (2,\sqrt{-6})\qquad\text{and}\qquad
	\p_3 = (3,\sqrt{-6}).
	$$
	Also, $\p_2^2 = 2\sO_K$ and $\p_3^2=3\sO_K$, so
	$\p_2$ and $\p_3$ define elements of order
	dividing $2$ in $\Cl(K)$.
	
	Is either $\p_2$ or $\p_3$ principal?  Fortunately,
	there is an easier norm trick that allows us to decide.
	Suppose $\p_2 = (\alpha)$, where $\alpha=a+b\sqrt{-6}$.
	Then
	$$
	2 = \Norm(\p_2)
	= \left|\Norm(\alpha)\right|
	= (a+b\sqrt{-6})(a-b\sqrt{-6})
	= a^2 + 6b^2.
	$$
	Trying the first few values of $a, b\in \ZZ$, we see that this
	equation has no solutions, so $\p_2$ can not
	be principal.  By a similar argument, we see that $\p_3$
	is not principal either.  Thus $\p_2$ and $\p_3$ define
	elements of order $2$ in $\Cl(K)$.
	
	Does the class of $\p_2$ equal the class of $\p_3$?
	Since $\p_2$ and $\p_3$ define classes of order~$2$,
	we can decide this by finding the class of $\p_2\cdot \p_3$.
	We have
	$$
	\p_2\cdot \p_3
	= (2,\sqrt{-6})\cdot (3,\sqrt{-6})
	= (6,2\sqrt{-6}, 3\sqrt{-6}) \subset (\sqrt{-6}).
	$$
	The ideals on both sides of the inclusion have norm $6$,
	so by multiplicativity of the norm, they must be the
	same ideal.  Thus $\p_2\cdot \p_3=(\sqrt{-6})$ is principal,
	which shows $\p_3$ is the inverse of $\p_2$ in $\Cl(K)$. But
	$\p_2$ had order $2$,
	so $\p_2$ and $\p_3$ represent the same element of $\Cl(K)$.
	We conclude that
	$$
	\Cl(K) = \langle \p_2 \rangle = \ZZ/2\ZZ.
	$$
\end{example}
