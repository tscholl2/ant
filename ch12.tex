%%%%%%%%%%%%%%%%%%%%%%%%%%%%%%%%%%%%%%%%%%%%%%%%%%%%%%%%%%%%%%%%%%%%%%%%%%
%% WEAKMW
%%%%%%%%%%%%%%%%%%%%%%%%%%%%%%%%%%%%%%%%%%%%%%%%%%%%%%%%%%%%%%%%%%%%%%%%%%

\chapter{The Weak Mordell-Weil Theorem}\label{ch:weakmw}

\section{Kummer Theory of Number Fields}\label{sec:kummernf}

Suppose $K$ is a number field and fix a positive integer~$n$.
Let $\mu_n$ denote the $n$th roots of unity in $\overline{K}$ as a group
under multiplication. Consider the exact sequence
\[
  1 \to \mu_n \to \overline{K}^* \xrightarrow{n} \overline{K}^* \to 1,
\]
where $n$ denotes the map $a\mapsto a^n$.

The corresponding long exact sequence from Theorem~\ref{thm:cohomology}
is
\[
  1 \to \mu_n(K) \to K^* \xrightarrow{n} K^*
  \to \H^1(K,\mu_n) \to \H^1(K,\overline{K}^*) =0,
\]
where $\mu_n(K)$ is the $n$th roots of unity contained in $K$.
The last equality follows from Theorem~\ref{thm:h90}.

Assume now that the group $\mu_n$ is contained in $K$.
Using Galois cohomology we obtain a relatively simple classification
of all abelian extensions of~$K$ with cyclic Galois group of order
dividing~$n$. Moreover, since the action of $\Gal(\overline{K}/K)$ on
$\mu_n$ is trivial, by our hypothesis that $\mu_n\subset K$,
Exercise~\ref{ex:H1hom} implies
\[
  \H^1(K,\mu_n) = \Hom(\Gal(\overline{K}/K),\mu_n).
\]
Thus we obtain an exact sequence
\[
  1 \to \mu_n \to K^* \xrightarrow{n} K^*
  \to \Hom(\Gal(\overline{K}/K),\mu_n) \to 1,
\]
or equivalently, an isomorphism
\[
  K^*/(K^*)^n \isom \Hom(\Gal(\overline{K}/K),\mu_n).
\]
By Galois theory, homomorphisms $\Gal(\overline{K}/K)\to \mu_n$ (up to
automorphisms of $\mu_n$) correspond to cyclic abelian extensions
of~$K$ with Galois group a subgroup of the cyclic group $\mu_n$.
Unwinding the definitions, this says that every
cyclic abelian extension of $K$ of degree dividing~$n$ is of the form
$K(a^{1/n})$ for some element $a\in K$.

One can prove via calculations that $K(a^{1/n})$ is unramified
outside $n$ and the primes that divide $\Norm(a)$.
Moreover, and this is a much bigger result, one can
combine this with facts about class groups and unit groups to prove
the following theorem:
\begin{theorem}\label{thm:maxunramfin}
  Suppose $K$ is a number field with $\mu_n\subset K$, where $n$
  is a positive integer. Let $L$ be the maximal extension of $K$
  such that
  \begin{enumerate}
    \item[(i)] $\Gal(L/K)$ is abelian,
    \item[(ii)] $n\cdot \Gal(L/K) = 0$, and
    \item[(iii)] $L$ is unramified outside a finite set~$S$ of primes.
  \end{enumerate}
  Then $L/K$ is of finite degree.
\end{theorem}
\begin{proof}[Sketch of Proof]
  Note that we may enlarge $S$ as needed. To see why,
  choose a finite set $S' \supseteq S$ and let $L'$ the
  maximal extension with respect to $S'$ as in the
  statement of the theorem. Because $L$ is unramified
  outside of $S$, it is certainly unramified outside of
  $S'$. By maximality of $L'$ this implies $L \subseteq L'$.
  Therefore it's sufficient to show the larger extension
  $L'/K$ is finite.
  
  We first argue that we can enlarge~$S$ so that the ring
  \[
    \sO_{K,S} =
    \{ a \in K^* \colon \Ord_\p(a\sO_K) \geq 0 \text{ for all } \p\not\in S \}
    \cup \{0\}
  \]
  is a principal ideal domain.
  One can show that for any~$S$, the ring $\overline{K}{K,S}$ is a Dedekind domain.
  \todo{possible exercise?}
  The condition $ \Ord_\p(a\sO_K) \geq 0$
  means that in the prime ideal factorization of the fractional ideal
  $a\sO_K$, we have that~$\p$ occurs to a nonnegative power. Thus we are
  allowing denominators at the primes in~$S$. Since the class group of
  $\sO_K$ is finite, there are primes $\p_1,\ldots, \p_r$ that generate
  the class group as a group (for example, take all primes with norm up to
  the Minkowski bound). Enlarge~$S$ to contain the primes $\p_i$.
  
  Note that we have used that \emph{the class group of $\sO_K$ is finite}.
  
  Next we want to show $\p_i\sO_{K,S}$ is the unit ideal. To see this,
  let $m$ be the order of $\p_i$ in the class group of $\sO_{K}$ so that
  $\p_i^m = (\alpha)$ for some $\alpha\in \sO_{K}$. Note the factorization
  of $\frac{1}{\alpha}\sO_{K}$ is $\p_i^{-m}$ so by construction
  $\frac{1}{\alpha}\in\sO_{K,S}.$ Since
  $\alpha\in \left(\p_i\sO_{K,S}\right)^m$ this shows $(\p_i\sO_{K,S})^m$
  is the unit ideal. It follows from the unique factorization of ideals
  in the Dedekind domain $\sO_{K,S}$ that $\p_i\sO_{K,S}$ is the unit ideal.
  
  Now we can show $\sO_{K,S}$ is a principal ideal domain. Let $\P$
  be a prime ideal of $\sO_{K,S}$. Since the $\p_i$ generate
  the class group of $\sO_K$, the restriction of $\P$ to $\sO_K$ is
  equivalent modulo a principal ideal to a product of the primes
  $\p_i$. Therefore $\P$ is equivalent modulo a principal ideal
  to a product of ideals of the form $\p_i\sO_{K,S}$. Because we showed
  $\p_i\sO_{K,S}$ was the unit ideal, this means $\P$ is principal.
  
  Next enlarge $S$ so that all primes over $n\sO_K$ are in~$S$.
  Note that $\sO_{K,S}$ is still a PID.  Let
  \[
    K(S,n) =
    \{ a \in K^*/(K^*)^n \colon n \mid \Ord_\p(a) \text{ for all } \p \not\in S \}.
  \]
  Then a refinement of the arguments at the beginning of
  this section show that $L$ is generated by all $n$th roots
  of the elements of $K(S,n)$ (specifically, their representatives in $K$).
  Thus it suffices to prove that $K(S,n)$ is finite.
  
  If $a \in \sO_{K,S}^*$ then $\Ord_\p(a)=0$ for all $\p \notin S$.
  So there is a natural map
  \[
    \phi: \sO_{K,S}^* \to K(S,n)
  \]
  sending $a$ to it's residue class in $K^*/(K^*)^n$.
  Suppose $a\in K^*$ is a representative of an element in $K(S,n)$.
  The ideal $a \sO_{K,S}$ has a factorization which is a product of $n$th
  powers, so it is an $n$th power of an ideal. Since $\sO_{K,S}$ is a PID,
  there is $b\in \sO_{K,S}$ and $u\in \sO_{K,S}^*$ such that
  \[
    a = b^n \cdot u.
  \]
  Thus $u\in \sO_{K,S}^*$ maps to $[a] \in K(S,n)$. This shows $\phi$
  is surjective.
  
  Recall {\em Dirichlet's unit theorem} (Theorem~\ref{thm:units}),
  which asserts that the group $\sO_K^*$ is a finitely generated
  abelian group of rank $r+s-1$.  More generally, we
  now show that $\sO_{K,S}^*$ is a finitely generated abelian group of
  rank $r+s+\#S -1$.
  Because we showed $\phi$ is surjective this would imply $K(S,n)$ is finitely generated.
  Since $K(S,n)$ is also a torsion group it must be finite which proves the theorem.
  
  \index{$S$-unit theorem}\index{Dirichlet!$S$-unit theorem}
  The fact that $\sO_{K,S}^*$ has rank $r+s-1 + \#S$ is sometimes
  referred to as the \emph{$S$-unit theorem} or the
  \emph{Dirichlet $S$-unit theorem}. To prove this theorem,
  let $\p_1,\ldots, \p_m$ be the primes in~$S$
  and define a map $\phi: \sO_{K,S}^* \to \ZZ^m$ by
  \[
    \phi(u) = (\Ord_{\p_1}(u), \ldots, \Ord_{\p_m}(u)).
  \]
  First we show that $\ker(\phi) = \sO_K^*$.  We have that
  $u\in \ker(\phi)$ if and only if $u \in \sO_{K,S}^*$
  and $\Ord_{\p_i}(u) = 0$ for all $i$; but the latter condition
  implies that $u$ is a unit at each prime in $S$. But
  $u\in \sO_{K,S}^*$ implies $\Ord_{\p}(u) = 0$ for all $\p\notin S$,
  so it follows that $\Ord_{\p}(u)=0$ for all primes $\p$ in $\sO_K$
  and therefore $u\in \sO_K^*$.
  Thus we have an exact sequence
  \[
    1 \to \sO_K^* \to \sO_{K,S}^* \xrightarrow{\phi} \ZZ^m.
  \]
  Next we show that the image of $\phi$ has finite index
  in $\ZZ^m$.  Let $h$ be the class number of $\sO_K$.
  For each $i$ there exists $\alpha_i \in \sO_K$
  such that $\p_i^h = (\alpha_i)$.  But $\alpha_i \in \sO_{K,S}^*$
  since $\Ord_{\p}(\alpha_i) = 0$ for all $\p\not\in S$ (by unique
  factorization). Then
  \[
    \phi(\alpha_i) = (0,\ldots,0,h,0,\ldots,0).
  \]
  It follows that $(h\ZZ)^m \subset \Im(\phi)$, so
  the image of $\phi$ has finite index in $\ZZ^m$.  It follows
  that $\sO_{K,S}^*$ has rank equal to $r+s-1+\#S$.
\end{proof}

\todo{TODO delete this comment?}
%\begin{remark}
%See Dannielle Li's excellent senior thesis for more details about this
%argument and the Mordell-Weil theorem.   Download it from\\
%{\tt http://modular.fas.harvard.edu/projects/}.
%\end{remark}

\section{Proof of the Weak Mordell-Weil Theorem}
Suppose $E$ is an elliptic curve over a number field~$K$, and
fix a positive integer~$n$.
Just as with number fields, we have an exact sequence
\[
  0 \to E[n] \to E \xrightarrow{n} E \to 0.
\]
Then we have an exact sequence
\[
  0 \to E[n](K) \to E(K) \xrightarrow{n} E(K) \to \H^1(K,E[n]) \to \H^1(K,E)[n] \to 0.
\]
Note the last term comes from replacing the codomain of
$\H^1(K,E[n]) \to \H^1(K,E)$ by the kernel of $\H^1(K,E) \xrightarrow{n} \H^1(K,E)$.
From this we obtain a short exact sequence
\begin{equation}\label{eqn:ke1}
  0 \to E(K)/n E(K) \to \H^1(K,E[n]) \to \H^1(K,E)[n] \to 0.
\end{equation}

Now assume, in analogy with Section~\ref{sec:kummernf}, that
$E[n]\subset E(K)$, i.e., all $n$-torsion points are defined over~$K$.
Then the Galois action on $E[n]$ is trivial so by
exercise~\ref{ex:H1hom} we have
\todo{TODO non-canonical iso?}
\[
  \H^1(K,E[n]) = \Hom(\Gal(\overline{K}/K),E[n]) \cong
  \Hom(\Gal(\overline{K}/K),(\ZZ/n\ZZ)^2),
\]
and the sequence (\ref{eqn:ke1}) induces an inclusion
\begin{equation}\label{eqn:injmw}
  E(K)/n E(K) \hookrightarrow \Hom(\Gal(\overline{K}/K),(\ZZ/n\ZZ)^2).
\end{equation}

Explicitly, this homomorphism sends a point $P$ to the homomorphism
defined as follows: Choose $Q \in E(\overline{K})$ such that $nQ = P$; then
send each $\sigma \in \Gal(\overline{K}/K)$ to $\sigma(Q)-Q\in E[n]$.

\begin{exercise}\label{ex:sigmaQ-QinEn}
  Consider the map $E(K) \to \Hom(\Gal(\overline{K}/K),E[n])$ defined above.
  First show this map is well defined, i.e., $\sigma(Q) - Q \in E[n]$
  for every $\sigma\in \Gal(\overline{K}/K)$.
  Then show it does not depend on the choice of $P$ modulo $nE(K)$
  so it indeed descends to a homomorphism on $E(K)/nE(K)$.
\end{exercise}

Because $E[n] \isom (\ZZ/n\ZZ)^2$, \todo{TODO non-canonical iso?}
given a point $P\in E(K)$, we obtain a homomorphism
$\varphi: \Gal(\overline{K}/K) \to (\ZZ/n\ZZ)^2$, whose kernel defines an
abelian extension~$L$ of $K$ that has exponent~$n$.
The amazing fact is that $L$ can be ramified only at the primes
of bad reduction for $E$ and the primes that divide~$n$.
Thus we can apply theorem~\ref{thm:maxunramfin} to see that there are
only finitely many such~$L$.

\begin{theorem}\label{thm:mwunram}
  Let $P\in E(K)$ and $L$ be the field obtained by adjoining the
  coordinates of all points $Q \in E(\overline{K})$ such that $nQ = P$.
  Then $L/K$ is unramified outside the set of primes
  dividing $n$ and primes of bad reduction for~$E$.
\end{theorem}
\begin{proof}[Sketch of Proof]
  This sketch closely follows \cite[Prop.~VIII.1.5b]{silverman:aec}.
  
  Fix a prime $\p$ of $K$ such that $\p \nmid n$ and $E$ has good
  reduction at $\p$. Let $\q$ be a prime of $L$ lying over $\p$.  
  Note that $\q$ is again a prime of good reduction for~$E$
  since we may use the same Weierstrass equation for $E$ as an
  elliptic curve over $L$.
  
  First one proves that for any extension $K'/K$ and any prime $\p'$ of $K'$
  such that $\p' \nmid n$ and $\p'$ is a prime of good reduction for~$E/K'$,
  the natural reduction map $\pi: E(K')[n] \to \tilde{E}(\sO_{K'}/\p')$
  is injective. The argument that~$\pi$ is injective uses
  \emph{formal groups}, whose development is outside the
  scope of this course.\footnote{For a proof using formal groups see 
    \cite[Prop.~VII.3.1b]{silverman:aec}.}
  
  Next, fix some $Q \in E[n]$ such that $nQ = P$.
  From Exercise~\ref{ex:sigmaQ-QinEn} we have that $\sigma(Q)-Q\in E[n]$ for
  all $\sigma \in \Gal(\overline{K}/K)$. Let $I_\q\subset \Gal(L/K)$
  be the inertia group for $\q/\p$. The action of $I_\q$ is trivial on 
  $\tilde{E}(\sO_L/\q)$ so for each $\sigma \in I_\q$ we have
  \[
    \pi(\sigma(Q) - Q) = \sigma(\pi(Q)) - \pi(Q) = \pi(Q) - \pi(Q) = 0.
  \]
  Since $\pi$ is injective, it follows that $\sigma(Q) = Q$ for $\sigma\in I_\q$,
  i.e., that $Q$ is fixed under $I_\q$. Repeating this argument for
  each $Q$ implies $I_\q$ is trivial and hence $\q/\p$ is unramified.
\end{proof}

\begin{theorem}[Weak Mordell-Weil]\label{thm:weakMW}
  Let $E$ be an elliptic curve over a number field~$K$, and
  let $n$ be any positive integer.  Then
  $E(K)/nE(K)$ is finitely generated.
\end{theorem}
\begin{proof}
  First suppose all elements of $E[n]$ have coordinates in~$K$.  Then
  the homomorphism (\ref{eqn:injmw}) provides an injection of $E(K)/n
  E(K)$ into
  \[
    \Hom(\Gal(\overline{K}/K), (\ZZ/n\ZZ)^2).
  \]
  By Theorem~\ref{thm:mwunram}, the image consists of homomorphisms whose
  kernels cut out an abelian extension of~$K$ unramified outside $n$
  and primes of bad reduction for~$E$.  Since this is a finite set of
  primes, Theorem~\ref{thm:maxunramfin} implies that the homomorphisms
  all factor through a finite quotient $\Gal(L/K)$ of $\Gal(\overline{\QQ}/K)$.
  Thus there can be only finitely many such homomorphisms, so the
  image of $E(K)/nE(K)$ is finite.  Thus $E(K)/nE(K)$ itself is
  finite, which proves the theorem in this case.
  
  Next suppose~$E$ is an elliptic curve over a number field, but do {\em
    not} make the hypothesis that the elements of $E[n]$ have
  coordinates in~$K$.  Since the group $E[n](\CC)$ is finite and its
  elements are defined over~$\overline{\QQ}$, the extension~$L$ of~$K$ got by
  adjoining to~$K$ all coordinates of elements of $E[n](\CC)$ is a finite
  extension.  It is also Galois, as we saw when constructing Galois
  representations attached to elliptic curves.
  By Proposition~\ref{prop:infres}, we have an exact sequence
  \[
    0 \to \H^1(L/K, E[n](L)) \to \H^1(K,E[n])\to \H^1(L,E[n]).
  \]
  The kernel of the restriction map
  $\H^1(K,E[n])\to \H^1(L,E[n])$ is finite, since it is
  isomorphic to the finite cohomology group
  $\H^1(L/K, E[n](L))$.  By the argument of the previous
  paragraph, the image of $E(K)/nE(K)$ in $\H^1(L,E[n])$
  under
  \[
    E(K)/n E(K) \hookrightarrow \H^1(K,E[n]) \xrightarrow{\Res} \H^1(L,E[n])
  \]
  is finite, since it is contained in the image of $E(L)/n E(L)$.
  Thus $E(K)/n E(K)$ is finite, since we just proved
  the kernel of $\Res$ is finite.
\end{proof}
